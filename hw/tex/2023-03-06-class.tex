\documentclass[12pt, leqno]{article}
\input{common}

\begin{document}
\hdr{2023-03-06}

Suppose $A \in \bbR^{n \times n}$ is symmetric, with eigenvalue decomposition 
$Q \Lambda Q^T$ where the eigenvalues are sorted in descending order of magnitude
and $|\lambda_1| > |\lambda_2|$.  If we write the eigenvector basis as
$Q = \begin{bmatrix} q_1 & Q_2 \end{bmatrix}$,
The cosine and sine of the acute angle between
$q_1$ and a unit vector $v$ are $\cos \angle(q_1,v) = |q_1^T v|$ and 
$\sin \angle(q_1,v) = \sqrt{1-|q_1^T v|^2} = \|Q_2^T v\|$.
Using these definitions, argue that when $v_k$ is the $k$th step of power
iteration,
\[
  \tan \angle(q_1,v_k) \leq |\lambda_2/\lambda_1|^k \tan \angle(q_1, v_0).
\]

\end{document}
