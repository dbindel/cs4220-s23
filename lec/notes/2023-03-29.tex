\documentclass[12pt, leqno]{article} %% use to set typesize
\input{common}

\usepackage{fontspec}
\usepackage{polyglossia}
\setmonofont{DejaVu Sans Mono}[Scale=MatchLowercase]
\usepackage[outputdir=pdf]{minted}

\providecommand{\tightlist}{%
  \setlength{\itemsep}{0pt}\setlength{\parskip}{0pt}}
\begin{document}



\hdr{2023-03-29}

\section{Fixed points and contraction mappings}

As discussed in previous lectures, many iterations we consider have the
form \[x^{k+1} = G(x^k)\] where
\(G : \mathbb{R}^n \rightarrow \mathbb{R}^n\). We call \(G\) a
\emph{contraction} on \(\Omega\) if it is Lipschitz with constant less
than one, i.e. \[\|G(x)-G(y)\| \leq \alpha \|x-y\|, \quad \alpha < 1.\]
A sufficient (but not necessary) condition for \(G\) to be Lipschitz on
\(\Omega\) is if \(G\) is differentiable and \(\|G'(x)\| \leq \alpha\)
for all \(x \in \Omega\).

According to the \emph{contraction mapping theorem} or \emph{Banach
fixed point theorem}, when \(G\) is a contraction on a closed set
\(\Omega\) and \(G(\Omega) \subset \Omega\), there is a unique fixed
point \(x^* \in \Omega\) (i.e.~a point such that \(x^* - G(x^*)\)). If
we can express the solution of a nonlinear equation as the fixed point
of a contraction mapping, we get two immediate benefits.

First, we know that a solution exists and is unique (at least, it is
unique within \(\Omega\)). This is a nontrivial advantage, as it is easy
to write nonlinear equations that have no solutions, or have continuous
families of solutions, without realizing that there is a problem.

Second, we have a numerical method -- albeit a potentially slow one --
for computing the fixed point. We take the fixed point iteration
\[x^{k+1} = G(x^k)\] started from some \(x^0 \in \Omega\), and we
subtract the fixed point equation \(x^* = G(x^*)\) to get an iteration
for \(e^k = x^k-x^*\): \[e^{k+1} = G(x^* + e^k) - G(x^*)\] Using
contractivity, we get
\[\|e^{k+1}\| = \|G(x^* + e^k) - G(x^*)\| \leq \alpha\|e^k\|\] which
implies that \(\|e^k\| \leq \alpha^k \|e^0\| \rightarrow 0.\)

When error goes down by a factor of \(\alpha > 0\) at each step, we say
the iteration is \emph{linearly convergent} (or \emph{geometrically
convergent}). The name reflects a semi-logarithmic plot of (log) error
versus iteration count; if the errors lie on a straight line, we have
linear convergence. Contractive fixed point iterations converge at least
linarly, but may converge more quickly.

\subsection{A toy example}

Consider the function \(G : \mathbb{R}^2 \rightarrow \mathbb{R}^2\)
given by
\[G(x) = \frac{1}{4} \begin{bmatrix} x_1-\cos(x_2) \\ x_2-\sin(x_1) \end{bmatrix}\]
This is a contraction mapping on all of \(\mathbb{R}^2\) (why?).

Let's look at $\|x^k-G(x^k)\|$ as a function of $k$, starting from the
initial guess $x^0 = 0$ (Figure~\ref{fig:toy-contraction-cvg}).

\begin{minted}{julia}
function test_toy_contraction()
	
    G(x) = 0.25*[x[1]-cos(x[2]); x[2]-sin(x[1])]
    
    # Run the fixed point iteration for 100 steps
    resid_norms = []
    x = zeros(2)
    for k = 1:100
	x = G(x)
	push!(resid_norms, norm(x-G(x), Inf))
    end
    
    x, resid_norms
end
\end{minted}

\begin{figure}
\begin{tikzpicture}
  \begin{semilogyaxis}[width=\textwidth,height=5cm,xlabel={$k$},ylabel={$\|x^k-G(x^k)\|$
}]
    \addplot table {data/2023-03-29-toy.txt};
  \end{semilogyaxis}
\end{tikzpicture}
\caption{Convergence for {\texttt test\_toy\_contraction}.}
\label{fig:toy-contraction-cvg}
\end{figure}

\subsubsection{Questions}

\begin{enumerate}
\def\labelenumi{\arabic{enumi}.}
\item
  Show that for the example above \(\|G'\|_\infty \leq \frac{1}{2}\)
  over all of \(\mathbb{R}^2\). This implies that
  \(\|G(x)-G(y)\|_\infty \leq \frac{1}{2} \|x-y\|_\infty\).
\item
  The mapping \(x \mapsto x/2\) is a contraction on \((0, \infty)\), but
  does not have a fixed point on that interval. Why does this not
  contradict the contraction mapping theorem?
\item
  For \(S > 0\), show that the mapping \(g(x) = \frac{1}{2} (x + S/x)\)
  is a contraction on the interval \([\sqrt{S}, \infty)\). What is the
  fixed point? What is the Lipschitz constant?
\end{enumerate}

\section{Newton's method for nonlinear equations}

The idea behind Newton's method is to approximate a nonlinear
\(f \in C^1\) by linearizations around successive guesses. We then get
the next guess by finding where the linearized approximaton is zero.
That is, we set
\[f(x^{k+1}) \approx f(x^k) + f'(x^k) (x^{k+1}-x^k) = 0,\] which we can
rearrange to \[x^{k+1} = x^k - f'(x^k)^{-1} f(x^k).\] Of course, we do
not actually want to form an inverse, and to set the stage for later
variations on the method, we also write the iteration as \begin{align*}
f'(x^k) p^k &= -f(x^k) \\
x^{k+1} &= x^k + p^k.
\end{align*}

\subsection{A toy example}

Consider the problem of finding the solutions to the system
\begin{align*}
  x + 2y &= 2 \\
  x^2 + 4y^2 &= 4.
\end{align*} That is, we are looking for the intersection of a straight
line and an ellipse. This is a simple enough problem that we can compute
the solution in closed form; there are intersections at \((0, 1)\) and
at \((2, 0)\). Suppose we did not know this, and instead wanted to solve
the system by Newton's iteration. To do this, we need to write the
problem as finding the zero of some function
\[f(x,y) = \begin{bmatrix} x + 2y - 2 \\ x^2 + 4y^2 - 4 \end{bmatrix} = 0.\]
We also need the Jacobian \(J = f'\):
\[\frac{\partial f}{\partial (x, y)} = 
\begin{bmatrix} 
  \frac{\partial f_1}{\partial x} & \frac{\partial f_1}{\partial y} \\
  \frac{\partial f_2}{\partial x} & \frac{\partial f_2}{\partial y}
\end{bmatrix}.\]

\begin{figure}
\begin{tikzpicture}
  \begin{semilogyaxis}[width=\textwidth,height=5cm,xlabel={$k$},ylabel={$\|f(x^k)\|$
}]
    \addplot table {data/2023-03-29-toy-newton.txt};
  \end{semilogyaxis}
\end{tikzpicture}
\caption{Convergence for {\texttt test\_toy\_newton} to $(0,1)$.}
\label{fig:toy-newton-cvg}
\end{figure}

We show the convergence of the Newton iteration in Figure~\ref{fig:toy-newton-cvg}.

\begin{minted}{julia}
function test_toy_newton(x, y)

    # Set up the function and the Jacobian
    f(x) = [x[1] + 2*x[2]-2; 
	    x[1]^2 + 4*x[2]^2 - 4]
    J(x) = [1      2     ;
	    2*x[1] 8*x[2]]
    
    # Run ten steps of Newton from the initial guess
    x = [x; y]
    fx = f(x)
    resids = zeros(10)
    for k = 1:10
	x -= J(x)\fx
	fx = f(x)
	resids[k] = norm(fx)
    end

    x, resids
end
\end{minted}

\subsubsection{Questions}

\begin{enumerate}
\def\labelenumi{\arabic{enumi}.}
\tightlist
\item
  Finding an (real) eigenvalue of \(A\) can be posed as a nonlinear
  equation solving problem: we want to find \(x\) and \(\lambda\) such
  that \(Ax = \lambda x\) and \(x^T x = 1\). Write a Newton iteration
  for this problem.
\end{enumerate}

\subsection{Superlinear convergence}

Suppose \(f(x^*) = 0\). Taylor expansion about \(x^k\) gives
\[0 = f(x^*) = f(x^k) + f'(x^k) (x^*-x^k) + r(x^k)\] where the remainder
term \(r(x^k)\) is \(o(\|x^k-x^*\|) = o(\|e^k\|)\). Hence
\[x^{k+1} = x^* + f'(x^{k})^{-1} r(x^k)\] and subtracting \(x^*\) from
both sides gives
\[e^{k+1} = f'(x^k)^{-1} r(x^k) = f'(x^k)^{-1} o(\|e^k\|)\] If
\(\|f'(x)^{-1}\|\) is bounded for \(x\) near \(x^*\) and \(x^0\) is
close enough, this is sufficient to guarantee \emph{superlinear
convergence}. When we have a stronger condition, such as \(f'\)
Lipschitz, we get \emph{quadratic convergence},
i.e.~\(e^{k+1} = O(\|e^k\|^2)\). Of course, this is all local theory --
we need a good initial guess!

\section{A more complex example}

We now consider a more serious example problem, a nonlinear system that
comes from a discretized PDE reaction-diffusion model describing (for
example) the steady state heat distribution in a medium going an
auto-catalytic reaction. The physics is that heat is generated in the
medium due to a reaction, with more heat where the temperature is
higher. The heat then diffuses through the medium, and the outside edges
of the domain are kept at the ambient temperature. The PDE and boundary
conditions are \begin{align*}
\frac{d^2 v}{dx^2} + \exp(v) &= 0, \quad x \in (0,1) \\
v(0) = v(1) &= 0.
\end{align*} We discretize the PDE for computer solution by introducing
a mesh \(x_i = ih\) for \(i = 0, \ldots, N+1\) and \(h = 1/(N+1)\); the
solution is approximated by \(v(x_i) \approx v_i\). We set
\(v_0 = v_{N+1} = 0\); when we refer to \(v\) without subscripts, we
mean the vector of entries \(v_1\) through \(v_N\). This discretization
leads to the nonlinear system
\[f_i(v) \equiv \frac{v_{i-1} - 2v_i + v_{i+1}}{h^2} + \exp(v_i) = 0.\]
This system has two solutions; physically, these correspond to stable
and unstable steady-state solutions of the time-dependent version of the
model.

\begin{figure}
\begin{tikzpicture}
  \begin{axis}[width=0.45\textwidth,height=5cm,xlabel={$x$},ylabel={$v(x)$},
      scaled y ticks=base 10:2]
    \addplot table {../data/2023-03-29-vplot0.txt};
  \end{axis}
\end{tikzpicture}
\hspace{0.08\textwidth}
\begin{tikzpicture}
  \begin{axis}[width=0.45\textwidth,height=5cm,xlabel={$x$},ylabel={$v(x)$}]
    \addplot table {../data/2023-03-29-vplot20.txt};
  \end{axis}
\end{tikzpicture}
\caption{Two solutions for the blow-up example.}
\label{fig:soln}
\end{figure}

We show the two solutions in Figure~\ref{fig:soln}.

To solve for a Newton step, we need both \(f\) and the Jacobian of \(f\)
with respect to the variables \(v_j\). This is a tridiagonal matrix,
which we can write as
\[J(v) = -h^{-2} T_N + \operatorname{diag}(\exp(v))\] where
\(T_N \in \mathbb{R}^{N \times N}\) is the tridiagonal matrix with \(2\)
down the main diagonal and \(-1\) on the off-diagonals.

\begin{minted}{julia}
function autocatalytic(v)
    N = length(v)
    fv = exp.(v)
    fv -= 2*(N+1)^2*v
    fv[1:N-1] += (N+1)^2*v[2:N  ]
    fv[2:N  ] += (N+1)^2*v[1:N-1]
    fv
end
\end{minted}

\begin{minted}{julia}
function Jautocatalytic(v)
    N = length(v)
    SymTridiagonal(exp.(v) .- 2*(N+1)^2, (N+1)^2 * ones(N-1))
end
\end{minted}

\begin{figure}
\begin{tikzpicture}
  \begin{semilogyaxis}[width=\textwidth,height=5cm,xlabel={$k$},ylabel={$\|f(v^k)\|$
}]
    \addplot table {../data/2023-03-29-rhist0.txt};
    \addplot table {../data/2023-03-29-rhist20.txt};
    \addplot table {../data/2023-03-29-rhist40.txt};
    \legend{$\alpha=0$, $\alpha=20$, $\alpha=40$};
  \end{semilogyaxis}
\end{tikzpicture}
\caption{Newton convergence for different starting guesses.}
\label{fig:newton-cvg}
\end{figure}

For an initial guess, we use $v_i = \alpha x_i(1-x_i)$ for different
values of $\alpha$.  For $\alpha = 0$, we converge to the stable
solution; for $\alpha = 20$ and $\alpha = 40$, we converge to the
unstable solution (Figure~\ref{fig:soln}).  We eventually see
quadratic convergence in all cases, but for $\alpha = 40$ there is a
longer period before convergence sets in
(Figure~\ref{fig:newton-cvg}).  For $\alpha = 60$, the method does not
converge at all.

\begin{minted}{julia}
function newton_autocatalytic(α, N=100, nsteps=50, rtol=1e-8; 
                              monitor=(v, resid)->nothing)
    v_all = [α*x*(1-x) for x in range(0.0, 1.0, length=N+2)]
    v = v_all[2:N+1]
    for step = 1:nsteps
	fv = autocatalytic(v)
	resid = norm(fv)
	monitor(v, resid)
	if resid < rtol
	    v_all[2:N+1] = v
	    return v_all
	end
	v -= Jautocatalytic(v)\fv
    end
    error("Newton did not converge after $nsteps steps")
end
\end{minted}

We can derive a Newton-like fixed point iteration from the observation
that if \(v\) remains modest, the Jacobian is pretty close to
\(-h^2 T_N\). This gives us the iteration
\[h^{-2} T_N v^{k+1} = \exp(v^k).\]

\begin{figure}
\begin{tikzpicture}
  \begin{semilogyaxis}[width=\textwidth,height=5cm,xlabel={$k$},ylabel={$\|f(v^k)\|$
}]
    \addplot table {../data/2023-03-29-rhist0.txt};
    \addplot table {../data/2023-03-29-rhistfp0.txt};
    \legend{Newton, Fixed point};
  \end{semilogyaxis}
\end{tikzpicture}
\caption{Newton convergence vs fixed point convergence.}
\label{fig:fp}
\end{figure}

In Figure~\ref{fig:fp}, we compare the convergence of this fixed point
iteration to Newton's method.  The fixed point iteration does
converge, but it shows the usual linear convergence, while Newton's
method converges quadratically.

\begin{minted}{julia}
function fp_autocatalytic(α, N=100, nsteps=500, rtol=1e-8;
			  monitor=(v, resid)->nothing)
    v_all = [α*x*(1-x) for x in range(0.0, 1.0, length=N+2)]
    v = v_all[2:N+1]
    TN = SymTridiagonal(2.0*ones(N), -ones(N-1))
    F = ldlt(TN)
    for step = 1:nsteps
	fv = autocatalytic(v)
	resid = norm(fv)
	monitor(v, resid)
	if resid < rtol
	    v_all[2:N+1] = v
	    return v_all
	end
	v[:] = F\(exp.(v)/(N+1)^2)
    end
    error("Fixed point iteration did not converge after $nsteps steps (α=$α)")
end
\end{minted}

\subsubsection{Questions}

\begin{enumerate}
\def\labelenumi{\arabic{enumi}.}
\item
  Consider choosing \(v = \alpha x (1-x)\) so that the equation is
  exactly satisfied at \(x = 1/2\). How would you do this numerically?
\item
  Modify the Newton solver for the discretization of the equation
  \(v'' + \lambda \exp(v) = 0\). What happens as \(\lambda\) grows
  greater than one? For what size \(\lambda\) can you get a solution?
  Try incrementally increasing \(\lambda\), using the final solution for
  the previous value of \(\lambda\) as the initial guess at the next
  value.
\end{enumerate}

\section{Some practical issues}

In general, there is no guarantee that a given solution of nonlinear
equations will have a solution; and if there is a solution, there is no
guarantee of uniqueness. This has a practical implication: many
incautious computationalists have been embarrassed to find that they
have ``solved'' a problem that was later proved to have no solution!

When we have reason to believe that a given system of equations has a
solution --- whether through mathematical argument or physical intuition
--- we still have the issue of finding a good enough initial estimate
that Newton's method will converge. In coming lectures, we will discuss
``globalization'' methods that expand the set of initial guesses for
which Newton's method converges; but globalization does not free us from
the responsibility of trying for a good guess. Finding a good guess
helps ensure that our methods will converge quickly, and to the
``correct'' solution (particularly when there are multiple possible
solutions).

We saw one explicit example of the role of the initial guess in our
analysis of the discretized blowup PDE problem. Another example occurs
when we use unguarded Newton iteration for optimization. Given a poor
choice of initial guess, we are as likely to converge to a saddle point
or a local maximum as to a minimum! But we will address this pathology
in our discussion of globalization methods.

If we have a nice problem and an adequate initial guess, Newton's
iteration can converge quite quickly. But even then, we still have to
think about when we will be satisfied with an approximate solution. A
robust solver should check a few possible termination criteria:

\begin{itemize}
\item
  \emph{Iteration count}: It makes sense to terminate (possibly with a
  diagnostic message) whenever an iteration starts to take more steps
  than one expects --- or perhaps more steps than one can afford. If
  nothing else, this is necessary to deal with issues like poor initial
  guesses.
\item
  \emph{Residual check}: We often declare completion when \(\|f(x^k)\|\)
  is sufficiently close to zero. What is ``close to zero'' depends on
  the scaling of the problem, so users of black box solvers are well
  advised to check that any default residual checks make sense for their
  problems.
\item
  \emph{Update check}: Once Newton starts converging, a good estimate
  for the error at step \(x^k\) is \(x^{k+1}-x^k\). A natural test is
  then to make sure that \(\|x^{k+1}-x^k\|/\|x^{k+1}\| < \tau\) for some
  tolerance \(\tau\). Of course, this is really an estimate of the
  relative error at step \(k\), but we will report the (presumably
  better) answer \(x^{k+1}\) -- like anyone else who can manage it,
  numerical analysts like to have their cake and eat it, too.
\end{itemize}

A common problem with many solvers is to make the termination criteria
too lax, so that a bad solution is accepted; or too conservative, so
that good solutions are never accepted.

One common mistake in coding Newton's method is to goof in computing the
Jacobian matrix. This error is not only very common, but also very often
overlooked. After all, a good approximation to the Jacobian often still
produces a convergent iteration; and when iterations diverge, it is hard
to distinguish between problems due to a bad Jacobian and problems due
to a bad initial guess. However, there is a simple clue to watch for
that can help alert the user to a bad Jacobian calculation. In most
cases, Newton converges quadratically, while ``almost Newton''
iterations converge linearly. If you think you have coded Newton's
method and a plot of the residuals shows linear behavior, look for
issues with your Jacobian code!


\end{document}
