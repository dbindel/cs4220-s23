\documentclass[12pt, leqno]{article} %% use to set typesize
\input{common}

\begin{document}
\hdr{2023-03-06}

\section{Power iteration}

In most introductory linear algebra classes, one computes
eigenvalues as roots of a characteristic polynomial.
For most problems, this is a {\em bad idea}: the roots of
the characteristic polynomial are often very sensitive to changes
in the polynomial coefficients even when they correspond to
well-conditioned eigenvalues.  Rather than starting from this
point, we will start with another idea: the {\em power iteration}.

Suppose $A \in \bbC^{n \times n}$ is diagonalizable, with eigenvalues
$\lambda_1, \ldots, \lambda_n$ ordered so that
\[
  |\lambda_1| \geq
  |\lambda_2| \geq \ldots \geq
  |\lambda_n|.
\]
Then we have $A = V \Lambda V^{-1}$ where
$\Lambda = \operatorname{diag}(\lambda_1, \ldots, \lambda_n)$.  Now, note that
\[
  A^k = (V \Lambda V^{-1})(V \Lambda V^{-1}) \ldots (V\Lambda V^{-1})
      = V \Lambda^k V^{-1},
\]
or, to put it differently,
\[
  A^k V = V \Lambda^k.
\]
Now, suppose we have a randomly chosen vector $x = V \tilde{x} \in \bbC^{n}$,
and consider
\[
  A^k x = A^k V \tilde{x} = V \Lambda^k \tilde{x}
        = \sum_{j=1}^n v_j \lambda_j^k \tilde{x}_j.
\]
If we pull out a constant factor from this expression, we have
\[
  A^k x = \lambda_1^k \left(
    \sum_{j=1}^n v_j \left( \frac{\lambda_j}{\lambda_1} \right)^k \tilde{x}_j
  \right).
\]
If $|\lambda_1| > |\lambda_2|$, then
$(\lambda_j/\lambda_1)^k \rightarrow 0$ for each $j > 1$, and for
large enough $k$, we expect $A^k x$ to be nearly parallel to $v_1$,
assuming $\tilde{x}_1 \neq 0$.  This is the idea behind the power
iteration:
\[
  x^{(k+1)} = \frac{A x^{(k)}}{\|A x^{(k)}\|}
           = \frac{A^k x^{(0)}}{\|A^k x^{(0)}\|}.
\]
Assuming that the first component of $V^{-1} x^{(0)}$ is nonzero
and that $|\lambda_1| > |\lambda_2|$, the iterates $x^{(k)}$
converge linearly to the ``dominant'' eigenvector of $A$, with the
error asymptotically decreasing by a factor of $|\lambda_1|/|\lambda_2|$
at each step.

There are three obvious potential problems with the power method:
\begin{enumerate}
\item
  What if the first component of $V^{-1} x^{(0)}$ is zero?
\item
  What $\lambda_1/\lambda_2$ is near one?
\item
  What if we want the eigenpair $(\lambda_j, v_j)$ for $j \neq 1$?
\end{enumerate}
The first point turns out to be a non-issue: if we choose $x^{(0)}$ at
random, then the first component of $V^{-1} x^{(0)}$ will be nonzero
with probability 1.  Even if we were so extraordinarily unlucky as to
choose a starting vector for which $V^{-1} x^{(0)}$ {\em did} have a
zero leading coefficient, perturbations due to floating point
arithmetic would generally bump us to the case in which we had a
nonzero coefficient.

The second and third points turn out to be more interesting,
and we address them now.


\section{Spectral transformation and shift-invert}

Suppose again that $A$ is diagonalizable with $A = V \Lambda V^{-1}$.
The power iteration relies on the identity
\[
  A^k = V \Lambda^k V^{-1}.
\]
Now, suppose that $f(z)$ is any function that is defined locally by a
convergent power series.  Then as long as the eigenvalues are within
the radius of convergence, we can define $f(A)$ via the same power series,
and
\[
  f(A) = V f(\Lambda) V^{-1}
\]
where $f(\Lambda) = \operatorname{diag}(f(\lambda_1), f(\lambda_2),
\ldots, f(\lambda_n))$.  So the spectrum of $f(A)$ is the image of
the spectrum of $A$ under the mapping $f$, a fact known as the
{\em spectral mapping theorem}.

As a particular instance, consider the function $f(z) = (z-\sigma)^{-1}$.
This gives us
\[
  (A-\sigma I)^{-1} = V (\Lambda - \sigma I)^{-1} V^{-1},
\]
and so if we run power iteration on $(A-\sigma I)^{-1}$, we will converge
to the eigenvector corresponding to the eigenvalue $\lambda_j$ for
which $(\lambda_j-\sigma)^{-1}$ is maximal --- that is, we find the eigenvalue
closest to $\sigma$ in the complex plane.  Running the power method on
$(A-\sigma I)^{-1}$ is sometimes called the shift-invert power method.


\section{Changing shifts}

If we know a shift $\sigma$ that is close to a desired eigenvalue,
the shift-invert power method may be a reasonable method.  But even
with a good choice of shift, this method converges at best linearly
(i.e. the error goes down by a constant factor at each step).
We can do better by choosing a shift {\em dynamically}, so that as
we improve the eigenvector, we also get a more accurate shift.

Suppose $\hat{v}$ is an approximate eigenvector for $A$, i.e.
we can find some $\hat{\lambda}$ so that
\begin{equation} \label{rq-deriv-approx}
  A \hat{v} - \hat{v} \hat{\lambda} \approx 0.
\end{equation}
The choice of corresponding approximate eigenvalues is not so clear,
but a reasonable choice (which is always well-defined when $\hat{v}$
is nonzero) comes from multiplying (\ref{rq-deriv-approx}) by $\hat{v}^*$
and changing the $\approx$ to an equal sign:
\[
  \hat{v}^* A \hat{v} - \hat{v}^* \hat{v} \hat{\lambda} = 0.
\]
The resulting eigenvalue approximation $\hat{\lambda}$ is
the {\em Rayleigh quotient}:
\[
  \hat{\lambda} = \frac{ \hat{v}^* A \hat{v} }{ \hat{v}^* \hat{v} }.
\]

If we dynamically choose shifts for shift-invert steps using
Rayleigh quotients, we get the {\em Rayleigh quotient iteration}:
\begin{align*}
  \lambda_{k+1} &= \frac{v^{(k)\,*} A v^{(k)}}{v^{(k)\,*} v^{(k)}} \\
  v^{(k+1)} &=
    \frac{ (A-\lambda_{k+1})^{-1} v^{(k)} }
         {\| (A-\lambda_{k+1})^{-1} v^{(k)} \|_2}
\end{align*}
Unlike the power method, the Rayleigh quotient iteration has locally
quadratic convergence --- so once convergence sets in, the number of
correct digits roughly doubles from step to step.  We will return to
this method later when we discuss symmetric matrices, for which
the Rayleigh quotient iteration has locally {\em cubic} convergence.


\section{Subspaces and orthogonal iteration}

So far, we have still not really addressed the issue of dealing with
clustered eigenvalues.  For example, in power iteration, what should
we do if $\lambda_1$ and $\lambda_2$ are very close?  If the ratio
between the two eigenvalues is nearly one, we don't expect the power
method to converge quickly; and we are likely to not have at hand a
shift which is much closer to $\lambda_1$ than to $\lambda_2$, so
shift-invert power iteration might not help much.  In this case, we
might want to relax our question, and look for the invariant subspace
associated with $\lambda_1$ and $\lambda_2$ (and maybe more
eigenvalues if there are more of them clustered together with
$\lambda_1$) rather than looking for the eigenvector associated with
$\lambda_1$.  This is the idea behind {\em subspace iteration}.

In subspace iteration, rather than looking at $A^k x_0$ for some
initial vector $x_0$, we look at $\mathcal{V}_k = A^k \mathcal{V}_0$,
where $\mathcal{V}_0$ is some initial subspace.  If $\mathcal{V}_0$ is
a $p$-dimensional space, then under some mild assumptions the space
$\mathcal{V}_k$ will asymptotically converge to the $p$-dimensional
invariant subspace of $A$ associated with the $p$ eigenvalues of $A$
with largest modulus.  The analysis is basically the same as the
analysis for the power method.  In order to actually {\em compute},
though, we need bases for the subspaces $\mathcal{V}_k$.  Let us define
these bases by the recurrence
\[
  Q_{k+1} R_{k+1} = A Q_k
\]
where $Q_0$ is a matrix with $p$ orthonormal columns and
$Q_{k+1} R_{k+1}$ represents an economy QR decomposition.
This recurrence is called {\em orthogonal iteration}, since
the columns of $Q_{k+1}$ are an orthonormal basis for
the range space of $A Q_k$, and the span of $Q_k$ is the span of $A^k Q_0$.

Assuming there is a gap between $|\lambda_{p}|$ and $|\lambda_{p+1}|$,
orthogonal iteration will usually converge to an orthonormal basis
for the invariant subspace spanned by the first $p$ eigenvectors of $A$.
But it is interesting to look not only at the behavior of the subspace,
but also at the span of the individual eigenvectors.  For example,
notice that the first column $q_{k,1}$ of $Q_k$ satisfies the recurrence
\[
  q_{k+1,1} r_{k+1,11} = A q_{k,1},
\]
which means that the vectors $q_{k,1}$ evolve according to the power method!
So over time, we expect the first columns of the $Q_k$ to converge to the
dominant eigenvector.  Similarly, we expect the first two columns of $Q_k$
to converge to a basis for the dominant two-dimensional invariant subspace,
the first three columns to converge to the dominant three-dimensional
invariant subspace, and so on.  This observation suggests that we might be
able to get a complete list of nested invariant subspaces by letting the
initial $Q_0$ be some square matrix.  This is the basis for the workhorse
of nonsymmetric eigenvalue algorithms, the {\em QR method}, which we will
(briefly) describe next time.


\end{document}
