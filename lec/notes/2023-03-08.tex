\documentclass[12pt, leqno]{article} %% use to set typesize
\input{common}

%\newcommand{\uQ}{\underline{Q}}
%\newcommand{\uR}{\underline{R}}

\begin{document}
\hdr{2023-03-09 -- 2023-03-11}

\section{Orthogonal iteration to QR}

The {\em QR iteration} is the workhorse for solving the nonsymmetric
eigenvalue problem.  Unfortunately, while the iteration itself is
simple to write, the derivation sometimes appears to be a work of
black magic.  In fact, the QR iteration is essentially the subspace
iteration we have already seen, re-cast in a different form.

\begin{enumerate}
\item
  The orthogonal iteration $\uQ^{(k+1)} R^{(k)} = A \uQ^{(k)}$ is a
  generalization of the power method.  In fact, the first column of
  this iteration is {\em exactly} the power iteration.  In general,
  the first $p$ columns of $\uQ^{(k)}$ are converging to an orthonormal
  basis for a $p$-dimensional invariant subspace associated with the
  $p$ eigenvalues of $A$ with largest modulus (assuming that there
  aren't several eigenvalues with the same modulus to make this
  ambiguous).
\item
  If all the eigenvalues have different modulus, orthogonal iteration
  ultimately converges to the orthogonal factor in a Schur form
  \[
    AU = UT
  \]
  What about the $T$ factor?  Note that $T = U^* A U$, so a natural
  approximation to $T$ at step $k$ would be
  \[
    A^{(k)} = (\uQ^{(k)})^* A \uQ^{(k)},
  \]
  and from the definition of the subspace iteration, we have
  \[
    A^{(k)} = (\uQ^{(k)})^* \uQ^{(k+1)} R^{(k)} = Q^{(k)} R^{(k)},
  \]
  where $Q^{(k)} \equiv (\uQ^{(k)})^* \uQ^{(k+1)}$ is unitary.
\item
  Note that
  \[
    A^{(k+1)}
    = (\uQ^{(k+1)})^* A^{(k)} \uQ^{(k+1)}
    = (Q^{(k)})^* A^{(k)} Q^{(k)}
    = R^{(k)} Q^{(k)}.
  \]
  Thus, we can go from $A^{(k)}$ to $A^{(k+1)}$ directly without
  the orthogonal factors from subspace iteration, simply by computing
  \begin{align*}
    A^{(k)} &= Q^{(k)} R^{(k)} \\
    A^{(k+1)} &= R^{(k)} Q^{(k)}.
  \end{align*}
  This is the QR iteration.
\end{enumerate}

\section{Practical problems}

There are two major problems with the basic QR iteration:
\begin{enumerate}
\item
  Each step of the QR iteration requires a QR factorization, which is
  an $O(n^3)$ operation.  This is rather expensive, and even in the happy
  case where we might be able to get each eigenvalue with a constant
  number of steps, $O(n)$ total steps at a cost of $O(n^3)$ each gives
  us an $O(n^4)$ algorithm.  Given that everything else we have done
  so far costs only $O(n^3)$, an $O(n^4)$ cost for eigenvalue computation
  seems excessive.
\item
  Like the power iteration upon which it is based, the basic iteration
  converges linearly, and the rate of convergence is related to the
  ratios of the moduli of eigenvalues.  Convergence is slow when there
  are eigenvalues of nearly the same modulus, and nonexistent when
  there are eigenvalues with the same modulus.
\end{enumerate}
The latter problem is dealt with by introducting shifts into QR,
much like we did with inverse iteration.  It is an interesting topic,
but not one we will take up here.  Instead, we will focus on the first
problem, which leads us to several interesting intermediate factorizations.

\section{Woah, We're Halfway There}

The QR iteration maps upper Hessenberg
matrices to upper Hessenberg matrices, and this fact allows us to do
one QR sweep in $O(n^2)$ time.  So how do we reduce to upper
Hessenberg in the first place?  Or, for that matter, to
to other ``halfway there'' forms:
\begin{align*}
  A &= Q H Q^T \quad \mbox{where $Q$ is orthogonal and $H$ upper
    Hessenberg} \\
  A &= Q T Q^T \quad \mbox{where $Q$ is orthogonal, $T$ symmetric
    tridiagonal, and $A$ symmetric} \\
  A &= U B V^T \quad \mbox{where $U$ and $V$ are orthogonal and $B$ is
    upper bidiagonal}
\end{align*}
Unlike the SVD or various eigendecompositions, these forms can be
computed directly.  And in many cases, as we will discuss, they are
just as useful!

\section{Hessenberg via Householder}

Let's start with an over-ambitious goal\footnote{Ah, but a man's reach
  should exceed his grasp...}: can we directly compute the Schur form
by applying Householder transformations on the left and right of a
matrix?  We already secretly know that the answer is no, but trying
and failing will set us up neatly for computing a {\em Hessenberg form} ---
and for understanding why the Hessenberg form is in fact a natural
target.
A matrix $H$ is {\em upper Hessenberg} if it has nonzeros only in the
upper triangle and the first subdiagonal.  For example, the nonzero
structure of a 5-by-5 Hessenberg matrix is
\[
  \begin{bmatrix}
    \times & \times & \times & \times & \times \\
    \times & \times & \times & \times & \times \\
           & \times & \times & \times & \times \\
           &        & \times & \times & \times \\
           &        &        & \times & \times
  \end{bmatrix}.
\]
For any square matrix $A$, we can find a unitarily similar Hessenberg
matrix $H = Q^* A Q$ in $O(n^3)$ time (a topic for next time).
Because $H$ is similar to $A$, they have the same eigenvalues; but
as it turns out, the special structure of the Hessenberg matrix
makes it possible to run QR in $O(n^2)$ per iteration.

How might we go about trying to compute a Schur form directly?  A
natural first step is to apply an orthogonal transformation from the
left in order to introduce zeros in the first column, and then apply
the same transformation on the right.  Let's try this for the $4
\times 4$ matrix case (we'll mark with a star each nonzero element
that is updated by the previous transformation):
\[
\begin{bmatrix}
  \times & \times & \times & \times \\
  \times & \times & \times & \times \\
  \times & \times & \times & \times \\
  \times & \times & \times & \times
\end{bmatrix} \mapsto
\begin{bmatrix}
  * & * & * & * \\
  0 & * & * & * \\
  0 & * & * & * \\
  0 & * & * & *
\end{bmatrix} \mapsto
\begin{bmatrix}
  * & * & * & * \\
  * & * & * & * \\
  * & * & * & * \\
  * & * & * & *
\end{bmatrix}.
\]
Oops!  Applying the transformation from the right scrambles the
columns, and undoes the zeros we started just introduced.  The
problem is that we got too greedy.  What if instead of trying to
zero out everything below the diagonal, we instead zero out everything
below the first subdiagonal?  Then we affect rows 2 through $n$,
and a subsequent transform affecting columns 2 through $n$ does not
kill the zeros we introduced:
\[
\begin{bmatrix}
  \times & \times & \times & \times \\
  \times & \times & \times & \times \\
  \times & \times & \times & \times \\
  \times & \times & \times & \times
\end{bmatrix} \mapsto
\begin{bmatrix}
  \times & \times & \times & \times \\
  * & * & * & * \\
  0 & * & * & * \\
  0 & * & * & *
\end{bmatrix} \mapsto
\begin{bmatrix}
  \times & * & * & * \\
  \times & * & * & * \\
  0 & * & * & * \\
  0 & * & * & *
\end{bmatrix}.
\]
A second step to introduce zeros in the second column finishes the
reduction to Hessenberg form:
\[
\begin{bmatrix}
  \times & \times & \times & \times \\
  \times & \times & \times & \times \\
  \times & \times & \times & \times \\
  \times & \times & \times & \times
\end{bmatrix} \mapsto
\begin{bmatrix}
  \times & \times & \times & \times \\
  \times & \times & \times & \times \\
  0 & * & * & * \\
  0 & 0 & * & *
\end{bmatrix} \mapsto
\begin{bmatrix}
  \times & \times & * & * \\
  \times & \times & * & * \\
  0 & \times & * & * \\
  0 & 0 & * & *
\end{bmatrix}.
\]
Let's make this concrete:
\begin{lstlisting}
function my_hessenberg(A)
	H = copy(A)
	n = size(A)[1]
	Q = Matrix{Float64}(I,n,n)

	for j = 1:n-2
		u = householder(H[j+1:n,j])
		H[j+1:n,j:n] -= 2*u*(u'*H[j+1:n,j:n])
		H[j+2:n,j]   .= 0.0
		H[:,j+1:n] -= 2*(H[:,j+1:n]*u)*u'
		Q[:,j+1:n] -= 2*(Q[:,j+1:n]*u)*u'
	end

	Q, H
end
\end{lstlisting}

\section{Hessenberg matrices and $O(n^2)$ QR steps}

The special structure of the Hessenberg matrix makes the Householder
QR routine very economical.  The Householder reflection computed in
order to introduce a zero in the $(j+1,j)$ entry needs only to operate
on rows $j$ and $j+1$.  Therefore, we have
\[
  Q^* H = W_{n-1} W_{n-2} \ldots W_1 H = R,
\]
where $W_{j}$ is a Householder reflection that operates only on rows
$j$ and $j+1$.  Computing $R$ costs $O(n^2)$ time, since each $W_j$
only affects two rows ($O(n)$ data).  Now, note that
\[
  R Q = R (W_1 W_2 \ldots W_{n-1});
\]
that is, $RQ$ is computed by an operation that first mixes the first
two columns, then the second two columns, and so on.  The only subdiagonal
entries that can be introduced in this process lie on the first subdiagonal,
and so $RQ$ is again a Hessenberg matrix.  Therefore, one step of QR iteration
on a Hessenberg matrix results in another Hessenberg matrix, and a Hessenberg
QR iteration step can be performed in $O(n^2)$ time.

As it happens, the Hessenberg QR step can be written in terms of {\em
  bulge chasing}.  The picture (in the 5-by-5 case) is as follows.  We
start with the original Hessemberg matrix, and first apply an
orthogonal transformation (the first in a QR factorization) to the
first two rows and then symmetrically to the first two columns.  This
introduces a nonzero (a ``bulge'') in the $(3,1)$ position.  Marking
the elements modified in the first step with a star, we have:
\[
  \begin{bmatrix}
    * & * & * & * & * \\
    * & * & * & * & * \\
    * & * & \times & \times & \times \\
           &        & \times & \times & \times \\
           &        &        & \times & \times
  \end{bmatrix}.
\]
From here, our goal is to ``chase'' the bulge out of the Hessenberg
structure.  We start by applying an orthogonal transformation to rows
2 and 3 to remove the $(3,1)$ element; applying the same
transformation to columns 2 and 3 introduces a bulge element in the
$(4,2)$ position; marking the newly modified elements with stars,
we have the new structure
\[
  \begin{bmatrix}
    \times & * & * & \times & \times \\
    * & * & * & * & * \\
    0 & * & * & * & * \\
           & * & * & \times & \times \\
           &        &        & \times & \times
  \end{bmatrix}.
\]
Continuing for two more step in a similar fashion, we have a nonzero
in the $(5,3)$ position, and then can restore the structure the rest
of the way to a Hessenberg form.

\section{Terrific Tridiagonals and Busy Bidiagonals}

When $A$ is a symmetric matrix, the corresponding Hessenberg form
is symmetric as well.  This means that all the entries
below the first subdiagonal are zero (by upper Hessenberg structure),
and so are all the entries above the first superdiagonal (by
symmetry).

What about bidiagonalization?  In bidiagonalization, we again
interleave transformations on the left and right, but now we allow
ourselves to use different transformations.  The key is that we
want to keep introducing new zero elements through these
transformations without destroying zeros we already created.
In the $4 \times 4$ case, the first two steps look like
\[
\begin{bmatrix}
  \times & \times & \times & \times \\
  \times & \times & \times & \times \\
  \times & \times & \times & \times \\
  \times & \times & \times & \times
\end{bmatrix} \mapsto
\begin{bmatrix}
  * & * & * & * \\
  0 & * & * & * \\
  0 & * & * & * \\
  0 & * & * & *
\end{bmatrix} \mapsto
\begin{bmatrix}
  \times & * & 0 & 0 \\
  0 & * & * & * \\
  0 & * & * & * \\
  0 & * & * & *
\end{bmatrix}
\]

The code is
\begin{lstlisting}
function my_bidiagonalization(A)
	B = copy(A)
	n = size(A)[1]
	U = Matrix{Float64}(I,n,n)
	V = Matrix{Float64}(I,n,n)

	for j = 1:n-1

		# Transform from the left to introduce zeros in column j
		u = householder(B[j:n,j])
		B[j:n,j:n] -= 2*u*(u'*B[j:n,j:n])
		B[j+1:n,j] .= 0.0
		U[:,j:n] -= 2*(U[:,j:n]*u)*u'

		# Transform from the right to introduce zeros in row j
		v = householder(B[j,j+1:n])
		B[j:n,j+1:n] -= 2*(B[j:n,j+1:n]*v)*v'
		B[j,j+2:n] .= 0.0
		V[:,j+1:n] -= 2*(V[:,j+1:n]*v)*v'

	end

	B, U, V
end
\end{lstlisting}

\section{Using the Factorizations}

Let's look concretely at two cases where these factorizations are
useful.  The first is in computing {\em transfer functions}
that occur in control theory, where we have the form
\[
  T(s) = c^T (sI-A)^{-1} b + d
\]
for various values of $s$.  Naively, it looks like we might have
to spend $O(n^3)$ per value of $s$ where we want the transfer
function; but we can instead use the Hessenberg reduction
$A = Q H Q^T$ to get
\[
  T(s) = (c^T Q) (sI-H)^{-1} (Q^T b) + d
\]
Solving a Hessenberg linear system like $(sI-H)$ can be done in
$O(n^2)$ time rather than $O(n^3)$ time (why?); in fact, this is one
of the structures that MATLAB's sparse solver checks for.

A second application again deals with parameter-dependent
systems, but in a different setting.  Suppose $A = UBV^T$ is a
bidiagonal reduction of $A$, and we are interested in computing
the Tikhonov-regularized solution for different values of the
regularization parameter.  How can we do this efficiently?
Note that
\[
\left\|
\begin{bmatrix} A \\ \lambda I \end{bmatrix} x -
\begin{bmatrix} b \\ 0 \end{bmatrix}
\right\| =
\left\|
\begin{bmatrix}
  B \\
  \lambda I
\end{bmatrix}
(V^T x) -
\begin{bmatrix}
  U^T b \\
  0
\end{bmatrix}
\right\|
\]
The latter equations can be reduced back to bidiagonal form in $O(n)$
time (left as an exercise for the student!).  Therefore, after the
initial bidiagonal reduction, we can solve the least squares
problem for each new value of the regularization parameter $\lambda$
in $O(n)$ additional work.

\section{Enter Arnoldi}

When we discussed QR factorization, we started with Gram-Schmidt and
then moved to the Householder-based method.  In talking about
reduction to Hessenberg form, we started with the Householder approach
--- but there is something akin to Gram-Schmidt as well.  We can read
the basic idea by looking at the columns of the Hessenberg matrix
equation $A Q = Q H$:
\[
  A q_{j} = \sum_{k=1}^j q_j h_{jk} + h_{j,j+1} q_{j+1}.
\]
Rearranging, we have
\[
  q_{j+1} = \frac{1}{h_{j,j+1}} \left( A q_{j} - \sum_{k=1}^j q_k
  h_{kj} \right)
\]
where $h_{kj} = q_k^T A q_j$.  That is, the coefficients $H$ can be
exactly derived from using Gram-Schmidt orthgonalization to
orthonormalize $A q_j$ against $\{q_1, \ldots, q_j\}$ for each
successive $j$!

The beauty about using the Arnoldi procedure to reduce a matrix to
upper Hessenberg form is two fold: we can readily apply
the method to sparse matrices; and we can {\em stop early!}.
This latter fact is true of the Householder-based methods as well.
But in the case of the Arnoldi procedure, we will end up making
use of partial Hessenberg reduction to turn Arnoldi into a method of
approximating linear system solutions (GMRES) and a method of
approximating eigenpairs for large systems (the Arnoldi method).

When $A$ is symmetric, the Arnoldi procedure becomes the {\em Lanczos}
procedure.  Note that in this case, we {\em tridiagonalize} the
original matrix, which means at each step we need (in exact
arithmetic) to orthogonalize each successive $Aq_j$ against only two
other vectors.  This rather remarkable fact allows us to parley the
Lanczos iteration into an eigenvalue iteration as well as two truly
remarkable iterations for solving linear systems: MINRES and the
famous method of conjugate gradients (CG).  We will return to these
methods later.

\end{document}
