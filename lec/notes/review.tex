\documentclass[12pt, leqno]{article}
\input{common}

\newcommand{\calK}{\mathcal{K}}
\newcommand{\calR}{\mathcal{R}}

\begin{document}

\section{Introduction}

CS 4220/5223/Math 4260 teaches numerical methods for linear algebra,
nonlinear equation solving, and optimization.  We used some examples
from differential equations and function approximation, but these
topics are mostly left to CS 4210/Math 4250.  Ideally, you should now
understand some basic methods, well enough to choose methods suited to
your problems (and vice-versa).  You should also know how to tweak the
standard strategies for particular types of problems.

Like most of CS%
\footnote{And, according to Shrek, like ogres and onions.}, numerical
methods come in layers.  For example, solving a large, difficult
optimization problem or nonlinear systems is likely to involve
\begin{itemize}
  \item A {\em continuation strategy} to ``sneak up'' on the hard
    problem by solving a sequence of easier nonlinear problems.
  \item A {\em Newton iteration} (or a related iteration) to solve
    each of these nonlinear problems by repeatedly approximating them
    with linear ones.
  \item A {\em Krylov subspace iteration} to solve the large linear
    systems of equations that appear during Newton iteration.
  \item A {\em sparse matrix factorization} to solve a {\em
    preconditioner} problem that approximates the solution of the
    large linear systems without the cost of solving them directly.
  \item And {\em dense matrix factorizations} that solve dense
    blocks that appear in the course of factoring the sparse matrix.
\end{itemize}
Each layer involves choices about methods, parameters, or termination
criteria.  These are guided by a mix of general error analysis, an
understanding of complexity of operations, and problem-specific
concerns.  Putting together a full solver stack like this is beyond
the scope of a first course, but we have seen all these ingredients
this semester.  Through projects, and some of the special topics at
the end of the semester, we have also seen how these ideas come
together.

The point of these review notes is not to supplant earlier lecture
notes or the book, but to give a retrospective survey of ground
we've covered in the past couple months.  This is also an excuse
for me -- and you! -- to think about the types of problems I might
ask on a final exam.

\section{Overview}

The meat of this class is {\em factorizations} and {\em iterations}.

Factorization involves rewriting a matrix as a product of matrices
with some special structure; the big examples from the course are:
\begin{align*}
  PA &= LU          & \mbox{LU factorization / Gaussian elimination} \\
  A &= R^T R        & \mbox{Cholesky} \\
  A &= QR           & \mbox{QR factorization} \\
  A &= U \Sigma V^T & \mbox{Singular Value Decomposition (SVD)} \\
  A &= U T U^T      & \mbox{Schur factorization}
\end{align*}
We discussed the computation of the first three factorizations
in enough detail to implement something, and touched more lightly
on decompositions for the SVD and Schur factorization.  These
factorizations provide an efficient way to solve linear systems
and least squares problems, but can also be used for various
other purposes, from determinants to data compression.

We use iterations to solve nonlinear problems, and even for large
linear problems where factorizations are too expensive.  The chief
building block is {\em fixed point iterations} of the form
\[
  x_{k+1} = G(x_k).
\]
The most important fixed point iteration is Newton's iteration,
which plays a central role in nonlinear equation solving and
optimization.  Though Newton on its own only converges locally, and
though Newton steps may be too expensive, the Newton framework gives
us a way of reasoning about a variety of iterations.  Fixed point
iterations (stationary iterations) for linear systems, such as Jacobi
and Gauss-Seidel iteration, are also an important building block for
preconditioning modern Krylov subspace iterations such as GMRES and CG.

When we solve a problem numerically, we care about getting the answer
fast enough and right enough.  To understand the ``fast enough'' part,
we need to understand the cost of computing and using factorizations
and the rate of convergence of iterations.  To understand the ``right
enough'' part, we need to understand how errors are introduced into a
numerical computation through input error, roundoff, or termination of
iterations, and how those errors propogate.  Our standard strategy is
to relate forward error to the backward error or residual error (which
can often be bounded in the context of a particular algorithm or
termination criterion) via a condition number (which depends on the
problem).  The key tools here are Taylor expansion (usually just to
first order) and matrix and vector norm bounds.

\section{Background}

I assume intro courses in calculus and linear algebra, enough
programming coursework to write and debug simple Julia scripts, and
the magical ``sufficient mathematical maturity.''  But people
forget things, and some of the background needed for numerics
isn't always taught in intro courses.  So here are some things
you should know that you might not remember from earlier work.

\subsection{Linear algebra background}

In what follows, as in most of what I've done in class, I will mostly
stick with real vector spaces.

\paragraph{Vectors}
You should know a vector as:
\begin{itemize}
\item An object that can be scaled or added to other vectors.
\item A column of numbers, often stored sequentially
  in computer memory.
\end{itemize}
We often map between the two pictures using a basis.  For example,
a basis for the vector space of quadratic polynomials in one variable
is $\{1, x, x^2\}$; using this basis, we might concretely represent a
polynomial $1 + x^2/2$ in computer memory using the coefficient
vector
\[
  c = \begin{bmatrix} 1 \\ 0 \\ 0.5 \end{bmatrix}.
\]
In numerical linear algebra, we use column vectors more often than
row vectors, but both are important.  A row vector defines a linear
function over column vectors of the same length.  For example,
in our polynomial example, suppose we want the row vector
corresponding to evaluation at $-1$.  With respect to the power basis
$\{1, x, x^2\}$ for the polynomial space, that would give us the
row vector
\[
  w^T = \begin{bmatrix} 1 & -1 & 1 \end{bmatrix}
\]
Note that if $p(x) = 1+x^2/2$, then
\[
  p(-1) = 1 + (-1)^2/2
  = w^T c
  = \begin{bmatrix} 1 & -1 & 1 \end{bmatrix}
    \begin{bmatrix} 1 \\ 0 \\ 0.5 \end{bmatrix}.
\]

\paragraph{Vector norms and inner products}
A {\em norm} $\|\cdot\|$ measures vector lengths.  It is
positive definite, homogeneous, and sub-additive:
\begin{align*}
  \|v\| & \geq 0 \mbox{ and } \|v\| = 0 \mbox{ iff } v = 0 \\
  \|\alpha v\| &= |\alpha| \|v\| \\
  \|u+v\| & \leq \|u\| + \|v\|.
\end{align*}
The three most common vector norms we work with are the
Euclidean norm (aka the 2-norm), the $\infty$-norm (or max norm),
and the $1$-norm:
\begin{align*}
  \|v\|_2 &= \sqrt{\sum_j |v_j|^2} \\
  \|v\|_\infty &= \max_j |v_j| \\
  \|v\|_1 &= \sum_j |v_j|
\end{align*}
Many other norms can be related to one of these three norms.

An {\em inner product} $\langle \cdot, \cdot \rangle$
is a function from two vectors into the real
numbers (or complex numbers for an complex vector space).  It is
positive definite, linear in the first slot, and symmetric (or
Hermitian in the case of complex vectors); that is:
\begin{align*}
  \langle v, v \rangle & \geq 0 \mbox{ and }
  \langle v, v \rangle = 0 \mbox{ iff } v = 0 \\
%
  \langle \alpha u, w \rangle &= \alpha \langle u, w \rangle
  \mbox{ and }
  \langle u+v, w \rangle = \langle u, w \rangle + \langle v, w \rangle \\
%
  \langle u, v \rangle &= \overline{\langle v, u \rangle},
\end{align*}
where the overbar in the latter case corresponds to complex
conjugation.  Every inner product defines a corresponding norm
\[
  \|v\| = \sqrt{ \langle v, v \rangle}
\]
The inner product and the associated norm satisfy
the {\em Cauchy-Schwarz} inequality
\[
  \langle u, v \rangle \leq \|u\| \|v\|.
\]
The {\em standard inner product} on $\bbR^n$ is
\[
  x \cdot y = y^T x = \sum_{j=1}^n y_j x_j.
\]
But the standard inner product is not the only inner product,
just as the standard Euclidean norm is not the only norm.

\paragraph{Matrices}
You should know a matrix as:
\begin{itemize}
\item A representation of a linear map
\item An array of numbers, often stored sequentially in memory.
\end{itemize}

A matrix can {\em also} represent a bilinear function mapping
two vectors into the real numbers (or complex numbers for complex
vector spaces):
\[
  (v,w) \mapsto w^T A v.
\]
Symmetric matrices also represent {\em quadratic forms}
mapping vectors to real numbers
\[
  \phi(v) = v^T A v
\]
We say a symmetric matrix $A$ is {\em positive definite} if
the corresponding quadratic form is positive definite, i.e.
\[
  v^T A v \geq 0 \mbox{ with equality iff } v = 0.
\]

Many ``rookie mistakes'' in linear algebra involve forgetting
ways in which matrices differ from scalars:
\begin{itemize}
\item
  Not all matrices are square.
\item
  Not all matrices are invertible (even nonzero matrices can be
  singular).
\item
  Matrix multiplication is associative, but not commutative.
\end{itemize}
Don't forget these facts!

\paragraph{Block matrices}
We often partition matrices into submatrices of different
sizes.  For example, we might write
\[
  \begin{bmatrix}
    a_{11} & a_{12} & b_1 \\
    a_{21} & a_{22} & b_2 \\
    c_1 & c_2 & d
  \end{bmatrix} =
  \begin{bmatrix}
    A & b \\
    c^T & d
  \end{bmatrix}, \mbox{ where }
  A = \begin{bmatrix} a_{11} & a_{12} \\ a_{21} & a_{22} \end{bmatrix},
  b = \begin{bmatrix} b_1 \\ b_2 \end{bmatrix},
  c = \begin{bmatrix} c_1 \\ c_2 \end{bmatrix}.
\]
We can manipulate block matrices in much the same way we manipulate
ordinary matrices; we just need to remember that matrix multiplication
does not commute.

\paragraph{Matrix norms}
The matrices of a given size form a vector space, and we can define
a norm for such a vector space the same way we would for any other
vector space.  Usually, though, we want matrix norms that are compatible with
vector space norms (a ``submultiplicative norm''), i.e. something that
guarantees
\[
  \|Av\| \leq \|A\| \|v\|
\]
The most common choice is to use an {\em operator norm}:
\[
  \|A\| \equiv \sup_{\|v\| = 1} \|Av\|.
\]
The operator 1-norm and $\infty$ norm are easy to compute
\begin{align*}
  \|A\|_1 &= \max_j \sum_i |a_{ij}| \\
  \|A\|_\infty &= \max_i \sum_j |a_{ij}|
\end{align*}
The operator 2-norm is theoretically useful, but not so easily computed.

In addition to the operator norms, the {\em Frobenius norm} is a
common matrix norm choice:
\[
  \|A\|_F = \sqrt{ \sum_{i,j} |a_{ij}|^2}
\]

\paragraph{Matrix structure}
We considered many types of {\em structure} for matrices this
semester.  Some of these structures are what I think of as ``linear
algebra structures,'' such as symmetry, skew symmetry, orthogonality,
or low rank.  These are properties that reflect behaviors of an
operator or quadratic form that don't depend on the specific basis for
the vector space (or spaces) involved.  On the other hand, matrices
with special nonzero structure -- triangular, diagonal, banded,
Hessenberg, or sparse -- tend to lose those properties under any but a
very special change of basis.  But these nonzero structures or matrix
``shapes'' are very important computationally.

\paragraph{Matrix products}
Consider the matrix-vector product
\[
  y = Ax
\]
You probably first learned to compute this matrix product with
\[
  y_{i} = \sum_j a_{ij} x_j.
\]
But there are different ways to organize the sum depending on how
we want to think of the product.  We could say that $y_i$ is the
product of row $i$ of $A$ (written $A_{i,:}$) with $x$; or we could
say that $y$ is a linear combination of the columns of $A$,
with coefficients given by the elements of $x$.  Similarly,
consider the matrix product
\[
  C = A B.
\]
You probably first learned to compute this matrix product with
\[
  c_{ij} = \sum_k a_{ik} b_{kj}.
\]
But we can group and re-order each of these sums in different ways,
each of which gives us a different way of thinking about matrix
products:
\begin{align*}
  C_{ij} &= A_{i,:} B_{:,j} & \mbox{(inner product)} \\
  C_{i,:} &= A_{i,:} B & \mbox{(row-by-row)} \\
  C_{:,j} &= A B_{:,j} & \mbox{(column-by-column)} \\
  C &= \sum_k A_{:,k} B_{k,:} & \mbox{(outer product)}
\end{align*}
One can also think of organizing matrix multiplication around a
partitioning of the matrices into sub-blocks.  Indeed, this is how
tuned matrix multiplication libraries are organized.

\paragraph{Fast matrix products}
There are some types of matrices for which we can compute
matrix-vector products very quickly.  For example, if
$D$ is a diagonal matrix, then we can compute $Dx$ with one
multiply operation per element of $x$.  Similarly, if
$A = uv^T$ is a rank-one matrix, we can compute $Ax$ quickly
by recognizing that matrix multiplication is associative
\[
  Ax = (uv^T) x = u (v^T x).
\]
Thus, we can apply $A$ with one dot product (between $v$ and $x$)
and a scaling operation.

\paragraph{Singular values and eigenvalues}
A square matrix $A$ has an eigenvalue $\lambda$ and corresponding eigenvector
$v \neq 0$ if
\[
  Av = \lambda v.
\]
A matrix is {\em diagonalizable} if it has a complete basis of
eigenvectors $v_1, \ldots, v_n$; in this case, we write the
{\em eigendecomposition}
\[
  AV = V\Lambda
\]
where $V = \begin{bmatrix} v_1 & \ldots & v_n \end{bmatrix}$
and $\Lambda = \operatorname{diag}(\lambda_1, \lambda_2, \ldots, \lambda_n)$.
If a matrix is not diagonalizable, we cannot write the
eigendecomposition in this form (we need Jordan blocks and generalized
eigenvectors).  In general, even if the matrix $A$ is real and
diagonalizable, we may need to consider complex eigenvalues and eigenvectors.

A real {\em symmetric} matrix is always diagonalizable with real
eigenvalues, and has an orthonormal basis of eigenvectors $q_1,
\ldots, q_n$, so that we can write the eigendecomposition
\[
  A = Q \Lambda Q^T.
\]
For a nonsymmetric (and possibly rectangular) matrix, the natural
decomposition is often not the eigendecomposition, but
the {\em singular value decomposition}
\[
  A = U \Sigma V^T
\]
where $U$ and $V$ have orthonormal columns (the left and right {\em
  singular vectors}) and $\Sigma = \operatorname{diag}(\sigma_1,
\sigma_2, \ldots)$ is the matrix of {\em singular values}.
The singular values are non-negative; by convention, they should
be in ascending order.

\subsection{Calculus background}

\paragraph{Taylor approximation in 1D}
If $f : \bbR \rightarrow \bbR$ has $k$ continuous derivatives, then
Taylor's theorem with remainder is
\[
  f(x+z) = f(x) + f'(x) z + \ldots + \frac{1}{(k-1)!} f^{(k-1)}(x) +
           \frac{1}{k!} f^{(k)}(x+\xi)
\]
where $\xi \in [x, x+z]$.  We usually work with simple linear
approximations, i.e.
\[
  f(x+z) = f(x) + f'(x) z + O(z^2),
\]
though sometimes we will work with the quadratic approximation
\[
  f(x+z) = f(x) + f'(x) z + \frac{1}{2} f''(x) z^2 + O(z^3).
\]
In both of these, when say the error term $e(z)$ is $O(g(z))$, we mean
that for small enough $z$, there is some constant $C$ such that
\[
  |e(z)| \leq C g(z).
\]
We don't need to remember a library of Taylor expansions, but it is
useful to remember that for $|\alpha| < 1$, we have the geometric series
\[
  \sum_{j=0}^\infty \alpha^j = (1-\alpha)^{-1}.
\]

\paragraph{Taylor expansion in multiple dimensions}
In more than one space dimension, the basic picture of Taylor's
theorem remains the same.  If $f : \bbR^n \rightarrow \bbR^m$, then
\[
  f(x+z) = f(x) + f'(x) z + O(\|z\|^2)
\]
where $f'(x) \in \bbR^{m \times n}$ is the {\em Jacobian matrix}
at $x$.  If $\phi : \bbR^n \rightarrow \bbR$, then
\[
  \phi(x+z) = \phi(z) + \phi'(x) z + \frac{1}{2} z^T \phi''(z) z + O(\|z\|^3).
\]
The row vector $\phi'(x) \in \bbR^{1 \times n}$ is the derivative of
$\phi$, but we often work with the {\em gradient} $\nabla \phi(x) =
\phi'(x)^T$.  The {\em Hessian} matrix $\phi''(z)$ is the matrix of
second partial derivatives of $\phi$.  Going beyond second order
expansion of $\phi$ (or going beyond a first order expansion of $f$)
requires that we go beyond matrices and vectors to
work with tensors involving more than two indices.  For this class,
we're not going there.

\paragraph{Variational notation}
A {\em directional derivative} of a function $f : \bbR^n \rightarrow
\bbR^m$ in the direction $u$ is
\[
  \frac{\partial f}{\partial u}(x) \equiv
  \left. \frac{d}{ds} \right|_{s=0} f(x+su) = f'(x) u.
\]
A nice notational convention, sometimes called {\em variational}
notation (as in ``calculus of variations'') is to write
\[
  \delta f = f'(x) \delta u,
\]
where $\delta$ should be interpreted as ``first order change in.''
In introductory calculus classes, this sometimes is called
a total derivative or total differential, though there one usually
uses $d$ rather than $\delta$.  There is a good reason for using
$\delta$ in the calculus of variations, though, so that's typically
what I do.

Variational notation can tremendously simplify the calculus
book-keeping for dealing with multivariate functions.  For example,
consider the problem of differentiating $A^{-1}$ with respect to
every element of $A$.  I would compute this by thinking of the
relation between a first-order change to $A^{-1}$ (written
$\delta [A^{-1}]$) and a corresponding first-order change to $A$
(written $\delta A$).  Using the product rule and differentiating
the relation $I = A^{-1} A$, we have
\[
  0 = \delta [A^{-1} A] = \delta [A^{-1}] A + A^{-1} \delta A.
\]
Rearranging a bit gives
\[
  \delta [A^{-1}] = -A^{-1} [\delta A] A^{-1}.
\]
One {\em can} do this computation element by element, but it's harder
to do it without the computation becoming horrible.

\paragraph{Matrix calculus rules}
There are some basic calculus rules for expressions involving matrices
and vectors that are easiest to just remember.  These are naturally
analogous to the rules in 1D.  If $f$ and $g$ are differentiable maps
whose composition makes sense, the multivariate chain rule says
\[
  \delta [f(g(x))] = f'(g(x)) \delta g, \quad
  \delta g = g'(x) \delta x
\]
If $A$ and $B$ are matrix-valued functions, we also have
\begin{align*}
  \delta [A+B] &= \delta A + \delta B \\
  \delta [AB] &= [\delta A] B + A [\delta B], \\
  \delta [A^{-1} B] &= -A^{-1} [\delta A] A^{-1} B + A^{-1} \delta B
\end{align*}
and so forth.  The big picture is that the rules of calculus work as
well for matrix-valued functions as for scalar-valued functions,
and the main changes account for the fact that matrix multiplication
does not commute.  You should be able to convince yourself of the
correctness of any of these rules using the component-by-component
reasoning that you most likely learned in an introductory calculus
class, but using variational notation (and the ideas of linear
algebra) simplifies life immensely.

A few other derivatives are worth having at your fingertips
(in each of the following formulas, $x$ is assumed variable
while $A$ and $b$ are constant
\begin{align*}
  \delta [Ax-b] &= A \delta x \\
  \delta [\|x\|^2] &= 2 x^T \delta x \\
  \delta \left[\frac{1}{2} x^T A x - x^T b\right] &= (\delta x)^T (Ax-b) \\
  \delta \left[\frac{1}{2} \|Ax-b\|^2 \right] &= (A \delta x)^T (Ax-b)
\end{align*}
and if $f : \bbR^n \rightarrow \bbR^n$ is given by $f_i(x) = \phi(x_i)$,
then
\[
  \delta [f(x)] = \operatorname{diag}(\phi'(x_1), \ldots, \phi'(x_n))
  \, \delta x.
\]


\subsection{CS background}

\paragraph{Order notation and performance}
Just as we use big-O notation in calculus to denote a function
(usually an error term) that goes to zero at a controlled rate as the
argument goes to zero, we use big-O notation in algorithm analysis to
denote a function (usually run time or memory usage) that grows at a
controlled rate as the argument goes to infinity.  For instance,
if we say that computing the dot product of two length $n$ vectors
is an $O(n)$ operation, we mean that the time to compute the dot
products of length greater than some fixed constant $n_0$ is bounded
by $C n$ for some constant $C$.  The point of this sort of analysis
is to understand how various algorithms scale with problem size
without worrying about all the details of implementation and
architecture (which essentially affect the constant $C$).

Most of the major factorizations of {\em dense} numerical linear
algebra take $O(n^3)$ time when applied to square $n \times n$
matrices, though some building blocks (like multiplying a matrix
by a vector or scaling a vector) take $O(n^2)$ or $O(n)$ time.
We often write the algorithms for factorizations that take $O(n^3)$
time using block matrix notation so that we can build these
factorizations from a few well-tuned $O(n^3)$ building blocks,
the most important of which is matrix-matrix multiplication.

\paragraph{Graph theory and sparse matrices}
In {\em sparse} linear algebra, we consider matrices that can be
represented by fewer than $O(n^2)$ parameters.  That might mean most
of the elements are zero (e.g.~as in a diagonal matrix), or it might
mean that there is some other low-complexity way of representing the
matrix (e.g.~the matrix might be a rank-1 matrix that can be
represented as an outer product of two length $n$ vectors).  We
usually reserve the word ``sparse'' to mean matrices with few
nonzeros, but it is important to recognize that there are other
{\em data-sparse} matrices in the world.

The {\em graph} of a sparse matrix $A \in \bbR^{N \times N}$ consists
of a set of $N$ vertices $\mathcal{V} = \{1, 2, \ldots, N\}$ and a set
of edges $\mathcal{E} = \{(i,j) : a_{ij} \neq 0\}$.  While the cost of
general dense matrix operations usually depends only on the sizes of
the matrix involved, the cost of sparse matrix operations can be
highly dependent on the structure of the associated graph.

\subsection{Julia background}

\paragraph{Building matrices and vectors}
Julia gives us several standard matrix and vector construction functions.
\begin{lstlisting}
  I               # A UniformScaling object representing an identity
  Matrix(I,n,n)   # A dense matrix representation of an identity
  Z = zeros(n,n)  # n-by-n matrix of zeros
  b = rand(n)     # length n random vector (uniform)
  e = ones(n)     # length n vector of ones
  D = Diagonal(e) # Diagonal matrix object (specialized type)
  D = diagm(e)    # Dense matrix representation of a diagonal matrix
  e2 = diag(D)    # Extract a matrix diagonal
\end{lstlisting}

\paragraph{Concatenating matrices and vectors}
In addition to functions for constructing specific types of matrices
and vectors, Julia lets us put together matrices and vectors by
horizontal and vertical concatenation.  This works with
matrices just as well as with vectors!
\begin{lstlisting}
  x = [1; 2]      # Column vector
  y = [1 2]       # Row vector
  M = [1 2; 3 4]  # 2-by-2 matrix
  M = [I A]       # Horizontal matrix concatenation
\end{lstlisting}

\paragraph{Transpose and rearrangemenent}
Julia lets us rearrange the data inside a matrix or vector in
a variety of ways.  In addition to the usual transposition
operation, we can also do ``reshape'' operations that let us
interpret the same data layout in computer memory in different ways.
\begin{lstlisting}
  # Reshape A to a vector, then back to a matrix
  # Note: Julia is column-major
  avec = A[:]
  A = reshape(avec, n, n)
  
  A = A'        # Conjugate transpose
  A = transp(A) # Simple transpose

  idx = randperm(n)   # Random permutation of indices
  Ac = A[:,idx]       # Permute columns of A
  Ar = A[idx,:]       # Permute rows of A
  Ap = A[idx,idx]     # Permute rows and columns
\end{lstlisting}

\paragraph{Submatrices, diagonals, and triangles}
Julia lets us extract specific parts of a matrix, like the diagonal
entries or the upper or lower triangle.
\begin{lstlisting}
  A = randn(6,6);         # 6-by-6 random matrix
  A[1:3,1:3]              # Leading 3-by-3 submatrix
  A[1:2:end,:]            # Rows 1, 3, 5
  A[:,3:end]              # Columns 3-6
  
  Ad = diag(A)            # Diagonal of A (as vector)
  A1 = diag(A,1)          # First superdiagonal
  Au = UpperTriangular(A) # Upper triangle view
  Al = LowerTriangular(A) # Lower triangle view
\end{lstlisting}

\paragraph{Matrix and vector operations}
Julia provides a variety of {\em elementwise} operations as well as
linear algebraic operations.  To distinguish elementwise
multiplication or division from matrix multiplication and linear
solves or least squares, we put a dot in front of the elementwise
operations.
\begin{lstlisting}
  y = d.*x    # Elementwise multiplication of vectors/matrices
  y = x./d    # Elementwise division
  z = x + y   # Add vectors/matrices
  z = x .+ 1  # Add scalar to every element of a vector/matrix
  
  y = A*x     # Matrix times vector
  y = x'*A    # Vector times matrix
  C = A*B     # Matrix times matrix

  # Don't use inv!
  x = A\b     # Solve Ax = b *or* least squares
  y = b'/A    # Solve yA = b^T or least squares
\end{lstlisting}

\paragraph{Things best avoided}
There are few good reasons to compute explicit matrix inverses or
determinants in numerical computations.  Julia does provide these
operations.  But if you find yourself typing {\tt inv} or {\tt det} in
Julia, think long and hard.  Is there an alternate formulation?
Could you use the forward slash or backslash operations for solving a
linear system?

\subsection{Floating point}
Most floating point numbers are essentially
{\em normalized scientific notation}, but in binary.
A typical normalized number in double precision looks like
\[
  (1.b_1 b_2 b_3 \ldots b_{52})_2 \times 2^{e}
\]
where $b_1 \ldots b_{52}$ are 52 bits of the {\em significand}
that appear after the binary point.  In addition to the normalize
representations, IEEE floating point includes subnormal numbers
(the most important of which is zero) that cannot be represented
in normalized form; $\pm \infty$; and Not-a-Number (NaN), used
to represent the result of operations like $0/0$.

The rule for floating point is that ``basic'' operations
(addition, subtraction, multiplication, division, and square root)
should return the true result, correctly rounded.  So a Julia
statement
\begin{lstlisting}
  # Compute the sum of x and y (assuming they are exact)
  z = x + y
\end{lstlisting}
actually computes $\hat{z} = \operatorname{fl}(x+y)$ where
$\operatorname{fl}(\cdot)$ is the operator that maps real numbers to
the closest floating point representation.  For numbers that are in
the normalized range (i.e.~for which $\operatorname{fl}(z)$ is a
normalized floating point number), the relative error in approximating
$z$ by $\operatorname{fl}(z)$ is smaller in magnitude than machine
epsilon; for double precision, $\epsilon_{\mathrm{mach}} = 2^{-53}
\approx 1.1 \times 10^{-16}$; that is,
\[
  \hat{z} = z(1+\delta), \quad |\delta| \leq \epsilon_{\mathrm{mach}}.
\]
We can {\em model} the effects of roundoff on a computation by writing
a separate $\delta$ term for each arithmetic operation in Julia;
this is both incomplete (because it doesn't handle non-normalized
numbers properly) and imprecise (because there is more structure to
the errors than just the bound of machine epsilon).  Nonetheless,
this is a useful way to reason about roundoff when such reasoning
is needed.

\subsection{Sensitivity, conditioning, and types of error}

There are several different ways we can think about error.  The most
obvious is the {\em forward error}: how close is our approximate
answer to the correct answer?  One can also look at {\em backward
  error}: what is the smallest perturbation to the problem such that
our approximation is the true answer?  Or there is {\em residual
  error}: how much do we fail to satisfy the defining equations?

For each type of error, we have to decide whether we want to look at
the {\em absolute} error or the {\em relative} error.  For vector
quantities, we generally want the {\em normwise} absolute or relative
error, but often it's critical to choose norms wisely.  The
{\em condition number} for a problem is the relation between relative
errors in the input (e.g. the right hand side in a linear system of
equations) and relative errors in the output (e.g. the solution to a
linear system of equations).  Typically, we analyze the effect of
roundoff on numerical methods by showing that the method in floating
point is {\em backward stable} (i.e.~the effect of roundoffs lead to
an error that is bounded by some polynomial in the problem size
times $\macheps$) and separately trying to show that the problem is
{\em well-conditioned} (i.e. small backward error in the problem inputs
translates to small forward error in the problem outputs).

We are usually concerned with {\em first-order} error analysis,
i.e.~error analysis based on a linearized approximation to the true
problem.

\subsection{Problems}

\begin{enumerate}
\item
  Consider the mapping from quadratic polynomials to cubic polynomials
  given by $p(x) \mapsto x p(x)$.  With respect to the power basis
  $\{1, x, x^2, x^3\}$, what is the matrix associated with this
  mapping?
\item
  Consider the mapping from functions of the form
  $f(x,y) = c_1 + c_2 x + c_3 y$
  to values at $(x_1,y_1)$, $(x_2,y_2)$, and $(x_3,y_3)$.
  What is the associated matrix?  How would you set up a system
  of equations to compute the coefficient vector $c$ associated
  with a vector $b$ of function values at the three points?
\item
  Consider the $L^2$ inner product between quadratic polynomials
  on the interval $[-1,1]$:
  \[
    \langle p, q \rangle = \int_{-1}^1 p(x) q(x) \, dx
  \]
  If we write the polynomials in terms of the power basis
  $\{1, x, x^2\}$, what is the matrix associated with this inner
  product (i.e. the matrix $A$ such that $c_p^T A c_q = \langle p, q
  \rangle$ where $c_p$ and $c_q$ are the coefficient vectors for
  the two polynomials.
\item
  Consider the weighted max norm
  \[
    \|x\| = \max_{j} w_j |x_j|
  \]
  where $w_1, \ldots, w_n$ are positive weights.  For a square matrix
  $A$, what is the operator norm associated with this vector norm?
\item
  If $A$ is symmetric and positive definite, argue that the
  eigendecomposition is the same as the singular value decomposition.
\item
  Consider the block matrix
  \[
    M = \begin{bmatrix} A & B \\ B^T & D \end{bmatrix}
  \]
  where $A$ and $D$ are symmetric and positive definite.  Show that if
  \[
    \lambda_{\min}(A) \lambda_{\min}(D) \geq \|B\|_2^2
  \]
  then the matrix $M$ is symmetric and positive definite.
\item
  Suppose $D$ is a diagonal matrix such that $AD = DA$.  If
  $a_{ij} \neq 0$ for $i \neq j$, what can we say about $D$?
\item
  Convince yourself that the product of two upper triangular
  matrices is itself upper triangular.
\item
  Suppose $Q$ is a differentiable {\em orthogonal} matrix-valued
  function.  Show that $\delta Q = Q S$ where $S$ is skew-symmetric,
  i.e. $S = -S^T$.
\item
  Suppose $Ax = b$ and $(A+D) y = b$ where $A$ is invertible and $D$
  is relatively small.  Assuming we have a fast way to solve systems
  with $A$, give an algorithm to compute $y$ to within an error of
  $O(\|D\|^2)$ in terms of two linear systems involving $A$ and a
  diagonal scaling operation.
\item
  Suppose $r = b-A\hat{x}$ is the residual associated with an
  approximate solution $\hat{x}$.  The {\em maximum componentwise
    relative residual} is
  \[
    \max_i |r_i|/|b_i|.
  \]
  How can this be written in terms of a norm?
\end{enumerate}

\newpage
\section{Linear systems}

We start with
\[
  Ax = b
\]
where $A \in \bbR^{n \times n}$ is square and nonsingular.  We
initially consider {\em direct solvers} that compute $x$ in a
finite number of steps using a factorization of $A$.
  
\subsection{Sensitivity and conditioning of linear systems}

We care about the sensitivity of linear systems for two reasons.
First, we compute using floating point, and the standard analysis
of many numerical methods involves analyzing backward stability
(a property purely of the algorithm) together with conditioning
(a property purely of the problem).  Second, many problems inherit
error in the input data from measurements or from other computations,
and sensitivity analysis is needed to analyze how sensitive a
computation might be to these input errors.

In most of our error analysis, we assume that a standard norm
(the 1-norm, 2-norm, or max-norm) is a reasonable way to measure
the sizes of inputs and outputs.  But if different elements of
the input or output represent values with different units, the
problem might be {\em ill-scaled}.  For this reason, it usually
makes sense to scale the system before doing any error analysis
(or solves).

\paragraph{Matrix multiplication}
Suppose $A \in \bbR^{n \times n}$ is nonsingular, and consider the computation
\[
  y = Ax.
\]
Here we treat $x$ as an input and $y$ as an output.  The condition
number relates relative perturbations to the input to relative
perturbations to the output.  That is, given
\[
  \hat{y} = A \hat{x},
\]
we would like to compute a bound on $\|\hat{y}-y\|/\|y\|$ in terms of
$\|\hat{x}-x\|/\|x\|$.  For any consistent matrix norm,
\begin{align*}
  \|x\| &= \|A^{-1} y\| \leq \|A^{-1}\| \|y\| \\
  \|\hat{y}-y\| & = \|A(\hat{x}-x)\| \leq \|A\| \|\hat{x}-x\| 
\end{align*}
and therefore
\[
  \frac{\|\hat{y}-y\|}{\|y\|} \leq \kappa(A)
  \frac{\|\hat{x}-x\|}{\|x\|}, \quad
  \kappa(A) \equiv \|A\| \|A^{-1}\|.
\]
We call $\kappa(A)$ the {\em condition number with respect to multiplication}.

Another perspective is to consider perturbations not to $x$, but to
$A$:
\[
  \hat{y} = \hat{A} x, \quad \hat{A} = A + E
\]
In this case, we have
\[
  \|\hat{y}-y\| = \|E(\hat{x}-x)\| \leq \|E\| \|\hat{x}-x\| 
\]
and
\[
  \frac{\|\hat{y}-y\|}{\|y\|} \leq
  \kappa(A) \frac{\|E\|}{\|A\|},
\]
where $\kappa(A) = \|A\| \|A^{-1}\|$ as before.

\paragraph{Linear systems}
Now suppose $A \in \bbR^{n \times n}$ is nonsingular and consider the
linear solve
\[
  Ax = b.
\]
If $\hat{x}$ is an approximate solution, the corresponding residual is
\[
  r = b-A\hat{x}
\]
or, put differently
\[
  \hat{x} = A^{-1} (b + r).
\]
Using the sensitivity analysis for matrix multiplication, we have
\[
  \frac{\|\hat{x}-x\|}{\|x\|} \leq \kappa(A) \frac{\|r\|}{\|b\|}.
\]
We can also look at the sensitivity with respect to perturbations
to $A$.  Let $\hat{A} = A + E$.  Using the Taylor expansion of the
inverse about $A$, we have
\[
  \hat{A}^{-1} = A^{-1} - A^{-1} E A^{-1} + O(\|E\|^2).
\]
Therefore if $\hat{x} = \hat{A}^{-1} b$, we have
\[
  \hat{x}-x = -A^{-1} E x + O(\|E\|^2),
\]
and by manipulating norm bounds,
\[
  \frac{\|\hat{x}-x\|}{\|x\|}
  \leq \kappa(A) \frac{\|E\|}{\|A\|} + O(\|E\|^2).
\]

\paragraph{Geometry of ill-conditioning}
In the case of the matrix two-norm, we have
\[
  \kappa_2(A) \equiv \frac{\sigma_{\max}(A)}{\sigma_{\min}(A)}.
\]
where $\sigma_{\max}(A) = \|A\|$ and $\sigma_{\min}(A) = 1/\|A^{-1}\|$
are the largest and smallest singular values of $A$.
The two-norm condition number can be interpreted geometrically as the
ratio between the longest and the smallest axes of the elliptical region
\[
  \{Ax : \|x\| \leq 1\}.
\]

\subsection{Gaussian elimination}

We think of Gaussian elimination as an algorithm for factoring a
nonsingular matrix $A$ as
\[
  PA = LU
\]
where $P$ is a permutation, $L$ is unit lower triangular, and $U$ is
upper triangular.  Given such a factorization, we can solve systems
involving $A$ by forward and backward substitution involving $L$
and $U$.

The simplest case for Gaussian elimination is a (block) 2-by-2 system
in which no pivoting is required:
\[
A = \begin{bmatrix} A_{11} & A_{12} \\ A_{21} & A_{22} \end{bmatrix}
  = \begin{bmatrix} L_{11} & 0 \\ L_{21} & L_{22} \end{bmatrix}
    \begin{bmatrix} U_{11} & U_{12} \\ 0 & U_{22} \end{bmatrix}.
\]
Reading off the matrix products, we have
\begin{align*}
  L_{11} U_{11} &= A_{11} \\
  L_{11} U_{12} &= A_{12} \\
  L_{21} U_{11} &= A_{21} \\
  L_{22} U_{22} &= A_{22} - L_{21} U_{12}
               = A_{22} - A_{21} A_{11}^{-1} A_{12}.
\end{align*}
That is, we can compute the $L$ and $U$ by solving the smaller
subproblem of factoring $A_{11}$, then computing the off-diagonal
blocks via triangular solves, and then factoring the {\em Schur
  complement} $A_{22}-L_{21} U_{12}$.  Like matrix multiplication, we
can think of Gaussian elimination in several different ways, and
choosing the submatrices differently provides different strategies for
Gaussian elimination.

We can also think of Gaussian elimination as applying a sequence of
{\em Gauss transformations} (or {\em elementary transformations}).
This perspective is particularly useful when we think about analogous
algorithms based on orthogonal transformations (Givens rotations or
Householder reflections) which lead to methods for QR factorization.
However we think about the method, dense Gaussian elimination involves
three nested loops (like matrix multiplication) and takes $O(n^3)$
time.  Once we have a factorization, solving linear systems with it
takes $O(n^2)$ time.

In general, of course, pivoting may be needed.  The usual
{\em row pivoting} strategy guarantees that all the entries of $L$ below the
main diagonal are less than one.  Alternate pivoting strategies are
possible, and are particularly attractive in a sparse or parallel
settings (where the data motion associated with pivoting is annoyingly
expensive).

Gaussian elimination is {\em usually} backward stable, i.e.~the
computed $L$ and $U$ correspond to some $\hat{A}$ which is close to
$A$ in the sense of normwise relative error.  It is possible
to construct examples where the backward error grows terribly,
but these occur fairly rarely.

\subsection{LU and Cholesky}

If $A$ is symmetric and positive definite, we may prefer {\em
  Cholesky} factorization to Gaussian elimination.  The Cholesky
factorization is
\[
  A = R^T R
\]
where $R$ is upper triangular (sometimes this is also written $LL^T$
where $L$ is lower triangular).  The Cholesky factorization exists
and is nonsingular iff $A$ is positive definite (if $A$ is
semidefinite, a singular Cholesky factor may exist).  Attempting
to compute the Cholesky factorization is a standard method for testing
positive definiteness of a matrix.

As with Gaussian elimination, we can think of the factorization in
blocky form as
\[
  \begin{bmatrix}
    A_{11} & A_{12} \\
    A_{12}^T & A_{22}
  \end{bmatrix} =
  \begin{bmatrix}
    R_{11}^T & 0 \\
    R_{12}^T & R_{22}^T
  \end{bmatrix}
  \begin{bmatrix}
    R_{11} & R_{12} \\
    0 & R_{22}
  \end{bmatrix}
\]
This again leads to an algorithm in which we factor the $A_{11}$
block, use triangular solves to compute the off-diagonal block of the
Cholesky factor, and then form and factor a Schur complement matrix.
For an SPD matrix, Chokesky factorization is backward stable even
without pivoting.

The Cholesky factorization must succeed because every Schur complement
of a symmetric positive-definite matrix is again symmetric and
positive-definite.  The {\em strictly diagonally dominant} matrices
share a similar property, and can also safely be factored (via LU)
without pivoting.

\subsection{Sparse solvers}

If the matrix $A$ is large and sparse, we might consider using a
sparse direct factorization method.  Typically, sparse LU or
Cholesky look like
\[
  PAQ = LU \quad \mbox{ or } \quad QAQ^T = R^T R
\]
where the column permutation $Q$ (or symmetric permutation, in
the case of Cholesky) is chosen to minimize {\em fill}, and
the row permutation $P$ is chosen for stability.  We say a nonzero
element of $L$ or $U$ is a {\em fill element} if the corresponding
location in $A$ is zero.

The Cholesky factorization of a sparse SPD matrix involves {\em no}
fill if the corresponding graph is a tree and if the ordering
always places children before parents (a {\em bottom-up} ordering).
Matrices that look ``almost'' like trees can be efficiently dealt
with by sparse factorization methods.  It turns out that 2D meshes
are usually fine, 3D meshes get expensive fast, and most ``small
world'' graphs generate tremendous amounts of fill.

{\em Band} matrices are sparse matrices in which all the nonzero
elements are restricted to a narrow region around the diagonal.
Band matrices are sufficiently common and regular that they are
handled by libraries like LAPACK that are devoted to dense linear
algebra algorithms.  LAPACK does not deal with more general sparse
matrices.

\subsection{Iterative refinement}

Suppose $\hat{A} = LU$ where $\hat{A}$ is an ``okay'' approximation
of $A$.  Such a factorization might be computed by ignoring pivoting,
or by using lower-precision arithmetic at an intermediate stage,
or we might just have a case where Gaussian elimination with partial
pivoting is not quite backward stable.  In this case, we can ``clean
up'' an approximate solution by {\em iterative refinement}:
\begin{align*}
x_0 &= U^{-1} (L^{-1} b) \\
x_{k+1} &= x_k + U^{-1} (L^{-1} (b-Ax_k))
\end{align*}
The error $e_k = x_k-x$ satisfies the iteration
\[
  e_{k+1} = (I-\hat{A}^{-1} A) e_k,
\]
and iterative refinement converges quickly if
$\|I-\hat{A}^{-1} A\| \ll 1$.

\subsection{Julia backslash}

The Julia backslash operator \verb|A\b| does different actions based
on the type of {\tt A} (and of {\tt b}).  For triangular matrices, it
applies forward or backward substitution; for diagonal matrices, it
applies elementwise scaling; for general sparse or dense matrices it
applies sparse and dense Gaussian elimination; and for factorization
objects, it uses the factorization to solve the linear system.  In
general, it does ``the right thing'' given the information that you
provide via the type system.  If you are solving a linear system, you
should always use backslash instead of {\tt inv}.

One thing that Julia's backslash does {\em not} do is to see whether
you've already solved a linear system involving the matrix in
question.  If you want to re-use a factorization, you need to do so
yourself.  This typically looks something like
\begin{lstlisting}
  F = lu(A)         # O(n^3)
  x = F\b           # Solve Ax=b using the factoriation in O(n^2)
  y = F\d           # O(n^2) again
  # ...
\end{lstlisting}

\subsection{Problems}

\begin{enumerate}
\item
  Suppose $A$ is square and singular, and consider $y = Ax$.
  Show by example that a {\em finite} relative error in the input
  $x$ can lead to an {\em infinite} relative error in the output $y$.
\item
  Give a $2 \times 2$ example for which an $O(\macheps)$ normwise
  relative residual corresponds to a normwise relative error near one.
\item
  Show that $\kappa_2(A) = 1$ iff $A$ is a scalar multiple of an
  orthogonal matrix.
\item
  Suppose $M$ is the elementary transformation matrix
  \[
    M = \begin{bmatrix} 1 & 0 \\ m & I \end{bmatrix}.
  \]
  What is $M^{-1}$?
\item
  Compute the Cholesky factorization of the matrix
  \[
  A = \begin{bmatrix}
         4 & 2 \\
         2 & 9
      \end{bmatrix}
  \]
\item
  Consider the matrix
  \[
  \begin{bmatrix}
    D & u \\
    u^T & \alpha
  \end{bmatrix}
  \]
  where $D$ is diagonal with positive diagonal elements larger than the
  corresponding entries of $u$.  For what range of $\alpha$ must
  the matrix be positive definite?
\item
  If $A$ is symmetric and positive definite with Cholesky factor $R$,
  show that $\kappa_2(A) = \kappa_2(R)^2$ (note: use the SVD).
\item
  If $\hat{A} = LU = A+E$, show that iterative refinement with
  the computed LU factors satisfies
  \[
    \|e_{k+1}\| \leq
    \left( \kappa(\hat{A}) \frac{\|E\|}{\|\hat{A}\|} \right) \|e_k\|
  \]
\end{enumerate}

\newpage
\section{Least squares problems}

Consider the equations
\[
  Ax = b
\]
where $A \in \bbR^{m \times n}$ and $b \in \bbR^n$.  We typically
consider $m > n$.  The system of equations may be {\em
  over-determined}, i.e.~$b$ is not in the range of $A$.  In this
case, we usually instead solve the least squares problem
\[
  \mbox{minimize } \|Ax-b\|^2
\]
The system may also be {\em ill-posed},~i.e.~the columns of $A$ are
linearly dependent (or nearly linearly dependent).  These conditions
are not mutually exclusive.

When $n$ and $m$ are small and the matrix $A$ is dense, we can solve
either linear systems or least squares problems using a few standard
matrix factorizations: LU/Cholesky, QR, or SVD.  When $n$ and $m$
are large and sparse, we may be able to use a sparse direct solver
(assuming the graph of the matrix in question is ``nice'').
Otherwise, we may prefer an iterative solver.

\subsection{Sensitivity and conditioning of least squares}

Suppose $A \in \bbR^{m \times n}$ with $m > n$ has singular values
$\sigma_1 > \ldots > \sigma_n$.
The {\em condition number for least squares}
is $\kappa(A) = \sigma_1/\sigma_n$.  If $\hat{x}$ is
an approximate solution to the least squares problem with residual
$\hat{r} = b-\hat{A} x$, then
\[
  \frac{\|\hat{x}-x\|}{\|x\|} \leq \kappa(A) \frac{\|\hat{r}\|}{\|b\|}.
\]
This is extremely similar to one of the bounds we saw for linear
systems.  Of course, $\hat{r}$ will not necessarily be close to zero
even if $\hat{x}$ is close to $x$!

It's possible to compute the sensitivity to perturbations to the
matrix $A$ as well, but this is much messier than in the linear system
case.

\subsection{Normal equations}

The {\em normal equations} for minimizing $\|Ax-b\|$ are
\[
  A^T A x = A^T b.
\]
These equations exactly correspond to finding a stationary point
$\nabla \phi(x) = 0$ where
\[
  \phi(x) = \frac{1}{2} \|Ax-b\|^2.
\]
The equations are called the normal equations because they
can be written as
\[
  A^T r = 0, \quad r = b-Ax,
\]
i.e.~the residual at the solution is orthogonal (normal) to everything
in the range space of $A$.
  
The matrix $A^T A$ is sometimes called the {\em Gram matrix}.  It is
symmetric and positive definite (assuming that $A$ has full column
rank), but $\kappa(A^T A) = \kappa(A)^2$.  Hence, if the conditioning
of $A$ is a concern, we might not want to solve the normal equations
exactly.

\subsection{QR}

Given a matrix $A \in \bbR^{m \times n}$ with $m > n$, the {\em economy} QR
decomposition of $A$ is 
\[
  A = QR, \quad Q \in \bbR^{m \times n}, R \in \bbR^{n \times n}
\]
where $Q$ has orthonormal columns and $R$ is upper triangular.  In the
{\em full} QR decomposition, $Q$ is square and $R$ is a rectangular
upper triangular matrix.  The QR decomposition can be computed via the
Gram-Schmidt process applied to the columns of $A$, though this is not
backward stable and is not the preferred approach most of the time.
The QR decomposition can be computed in a backward stable fashion via
the {\em Householder QR} procedure, which applies $n$ simple
orthogonal transformations (Householder reflections of the form
$I-2uu^T$ where $\|u\| = 1$) that ``zero out'' the subdiagonal
elements in each of the $n$ columns in turn.

The QR decomposition is closely related to the normal equations
system: $R$ is the Cholesky factor of $A^T A$ (to within a
sign-rescaling of the diagonal of $R$) and $Q = A R^{-1}$ has
orthonormal columns.  But while computing $R$ by Cholesky
factorization of $A^T A$ involves a sub-problem with condition number
$\kappa(A)^2$, solving the system
\[
  R^T x = Q^T b
\]
involves just a solve with $R$, which has the same condition
number as $A$.

Solving a least squares problem via QR is moderately more expensive
than solving the Cholesky problem.  However, QR is somewhat more
numerically stable.

\subsection{SVD}

Just as we can solve the least squares problem with the economy QR
decomposition, we can also solve with the economy SVD $A = U \Sigma
V^T$:
\[
  x = V \Sigma^{-1} U^T b.
\]
If $A \in \bbR^{m \times n}$ has the economy QR decomposition
$A = QR$, we can compute the economy SVD of $A$ using the QR
decomposition together with the SVD of $R$.  If $m$ is sufficiently
larger than $n$, then most of the work goes into the QR step.

The SVD is even more numerically stable than QR decomposition,
but the primary reason for using the SVD is that we can analyze
the behavior of the singular values for reasoning about ill-posed
problems.

\subsection{Pseudo-inverses}

The process of solving a least squares problem is a linear operation,
which we write as
\[
  x = A^\dagger b.
\]
The symbol $A^\dagger$ is the (Moore-Penrose) {\em pseudoinverse},
which we expand as
\begin{align*}
  A^\dagger
  &= (A^T A)^{-1} A^T & \mbox{(Normal equations)} \\
  &= R^{-T} Q & \mbox{(QR)} \\
  &= V \Sigma^{-1} U^T & \mbox{(SVD)}
\end{align*}

\subsection{Ill-posed problems and regularization}

An {\em ill-posed} least squares problem is one in which the matrix
$A$ is ill-conditioned.  This means that there is a large set of
vectors $\hat{x}$ that explain the data equally well -- that is
$A(\hat{x}-x)$ is around the same order of magnitude as the
error in the measurement vector $b$.  Hence, the data is not good
enough to tell us which of the possible solution vectors is
really appropriate, and the usual least-squares approach
{\em overfits} the data, and if the coefficients $x$ are later used
to model behavior at a new data point, the prediction will be poor.

When the data does not provide enough information to fit a model,
we need to incorporate prior knowledge that is not based on the data.
This leads to {\em regularized least squares}.  Some common
approaches include
\begin{itemize}
\item Factor selection, i.e. predicting based on only a subset of
  the columns of $A$:
  \[
    \tilde{x}_{\mathcal{I}} = A_{:,\mathcal{I}}^\dagger b.
  \]
  The relevant subset $\mathcal{I}$ may be determined using QR with column
  pivoting or using more sophisticated heuristics based on an SVD.
\item Truncated SVD, i.e. computing
  \[
    \tilde{x} = V_{:,k} \Sigma_{1:k,1:k}^{-1} U_{:,k}^T b.
  \]
  This completely discards the influence of ``poorly-behaved'' directions.
\item Tikhonov regularization, i.e. minimizing
  \[
    \phi_{\mbox{Tik}}(x; \lambda) = \frac{1}{2} \|Ax-b\|^2 + \frac{\lambda^2}{2} \|x\|_M^2
  \]
  where $M$ is some positive definite matrix and $\lambda$ is a small
  regularization parameter.  The first term penalizes mismatch to the
  data; the second term penalizes overly large coefficients.
\end{itemize}
Each of these approaches has a parameter that controls the balance
between fitting the data and enforcing the assumptions.  For methods
based on subset selection or truncated SVD, one has to choose the
number of retained directions $k$; for Tikhonov regularization, one
has to choose the regularization parameter $\lambda$.  If something is
known in advance about the error, these parameters can be chosen a
priori.  Usually, though, one chooses the parameter in an adaptive
way based on some criterion.  Examples include the PRESS statistic,
corresponding to the sum of squared prediction errors in a
leave-one-out cross-validation process, or using the ``L-curve''
(topics we discussed briefly toward the end of the semester).

\subsection{Problems}

\begin{enumerate}
\item
  Suppose $M$ is symmetric and positive definite, so that
  $\|x\|_M = \sqrt{x^T M x}$ is a norm.  Write the normal
  equations for minimizing $\|Ax-b\|_M^2$.
\item
  Suppose $A \in \bbR^{n \times 1}$ is a vector of all ones.
  Show that $A^{\dagger} b$ is the sample mean of the entries
  of $b$.
\item
  Suppose $A = QR$ is an economy QR decomposition.
  Why is $\kappa(A) = \kappa(R)$?
\item
  Suppose we have economy QR decompositions for
  $A_1 \in \bbR^{m_1 \times n}$ and $A_2 \in \bbR^{m_2 \times n}$,
  i.e.
  \[
    A_1 = Q_1 R_1, \quad A_2 = Q_2 R_2
  \]
  Show that we can compute the QR decomposition of
  $A_1$ and $A_2$ stacked as
  \[
    \begin{bmatrix} A_1 \\ A_2 \end{bmatrix} = Q R, \quad
    Q = \begin{bmatrix} Q_1 \\ Q_2 \end{bmatrix} \tilde{Q}
  \]
  where
  \[
    \tilde{Q} R = \begin{bmatrix} R_1 \\ R_2 \end{bmatrix}
  \]
  is an economy QR decomposition.
\item
  Give an example of $A \in \bbR^{2 \times 1}$ and $b \in \bbR^2$
  such that a small relative change to $b$ results in a large
  relative change to the solution of the least squares problem.
  What is the condition number of $A$?
\item
  Write the normal equations for a Tikhonov-regularized least squares
  problem.
\item
  Show that $\Pi = A A^\dagger$ is a projection ($\Pi^2 = \Pi$) and
  that $\Pi b$ is the closest point to $b$ in the range of $A$.
\item
  Using the normal equations approach, find the coefficients
  $\alpha$ and $\beta$ that minimize
  \[
    \phi(\alpha, \beta) = \int_{-1}^1 (\alpha + \beta x - f(x))^2 \, dx
  \]
\end{enumerate}

\newpage
\section{Eigenvalues}

In this section, we discussed the eigenvalue problem
\[
  A x = \lambda x.
\]
Depending on the context, one might want all eigenvalues of $A$ or
only some; eigenvalues only, eigenvectors only, or both eigenvalues
and eigenvectors; row and column eigenvectors or only one or the
other.  Different methods give different information.

\subsection{Why eigenvalues?}

There are several reasons why we might want to compute
eigenvalues or eigenvectors
\begin{itemize}
\item
  Eigenvalue decompositions are often used to reason about systems of
  linear differential equations or difference equations.  Eigenvalues
  give information about how special solutions grow, decay, or
  oscillate; eigenvectors give the corresponding ``mode shapes''.
\item
  Eigenvalue problems involving tridiagonal matrices are common in
  the theory of special functions, and play an important role in
  numerical quadrature methods (for example).
\item
  Eigenvalue problems play a special role in linear control theory,
  and eigenvalue decompositions can be efficiently solve a variety of
  problems that arise there.  We gave one example (Sylvester
  equations) in lecture.
\item
  Symmetric eigenvalue problems are among the few {\em non-convex}
  optimization problems that we know how to reliably solve.  Many
  other optimization problems can be approximated by (relaxed to)
  eigenvalue problems.  This is the basis of spectral graph
  partitioning and spectral clustering, for example.
\item
  Eigenvalue finding and polynomial root finding are essentially
  equivalent.  For example, in the {\tt Polynomial.jl} package,
  the {\tt roots} command finds
  the roots of a polynomial via an equivalent eigenvalue problem.
\end{itemize}
There is also information that can be derived by eigenvalues {\em or}
by other methods.  Often an eigenvalue decomposition is useful for
analysis, and another approach is useful for computation.

\subsection{Jordan to Schur}

In an introductory class, you may have learned about the Jordan
canonical form.  For almost all matrices, we have a basis of
eigenvectors $V$ and can write
\[
  A V = V \Lambda.
\]
In some cases in which we have eigenvalues with high multiplicity,
we may need generalized eigenvectors, and replace the eigenvalue
matrix $\Lambda$ with a matrix that has eigenvalues on the diagonal
and some ones on the first superdiagonal.  However, the Jordan form
is {\em discontinuous} in the entries of the matrix; an
infinitesimally small perturbation can change one of the superdiagonal
elements from a zero to a one.  Also, the eigenvector matrix $V$ can
in general be rather poorly behaved (unless $A$ is symmetric or has
other special structure).

Numerical analysts prefer the {\em Schur form} to the Jordan form.
The (complex) Schur form is
\[
  A U = U T
\]
where $U \in \bbC^{n \times n}$ is a unitary matrix and
$T \in \bbC^{n \times n}$ is upper triangular.  The
real Schur form is
\[
  A Q = Q T
\]
where $Q \in \bbR^{n \times n}$ is an orthogonal matrix and $T \in
\bbR^{n \times n}$ is a block upper triangular matrix with $1 \times
1$ blocks (corresponding to real eigenvalues) and $2 \times 2$ blocks
(corresponding to complex conjugate pairs) on the diagonal.  As in the
Jordan form, the diagonal elements of $T$ in the complex Schur form
are the eigenvalues; but where the columns of $V$ in the Jordan form
correspond to eigenvectors (or generalized eigenvectors), the columns
of $U$ or $Q$ in the Schur factor form bases for a sequence of nested
invariant subspaces.  That is, for each $1 \leq k \leq n$, we have
\[
  A U_{:,1:k} = U_{1:k,:} T_{1:k,1:k}
\]
in the complex Schur form; and similarly for the real Schur form we
have
\[
  A Q_{:,1:k} = Q_{1:k,:} T_{1:k,1:k}
\]
for each $1 \leq k \leq n$ such that taking the first $k$ columns does
not split a $2 \times 2$ diagonal block.

If we insist, we can recover eigenvectors from the Schur form.
Consider the complex Schur form, and suppose we are interested
in the eigenvector associated with the eigenvalue $t_{kk}$
(which we will assume for simplicity has multiplicity 1).
Then solving the system
\[
  \left( T_{1:(k-1),1:(k-1)} - t_{kk} I \right) w + T_{1:(k-1),k} = 0
\]
gives us a vector
\[
  v = U_{:,(1:k-1)} w + U_{:,k}
\]
with the property that
\[
  A v
  = A U_{:,(1:k)} \begin{bmatrix} w \\ 1 \end{bmatrix}
  = U_{:,(1:k)} T \begin{bmatrix} w \\ 1 \end{bmatrix}
  = U_{:,(1:k)} t_{kk} \begin{bmatrix} w \\ 1 \end{bmatrix}
  = v t_{kk}.
\]
Hence, computing eigenvectors from the Schur form can be done
at the cost of one triangular solve per eigenvector.

We can go the other way as well: given a Jordan form, we can
easily compute the corresponding Schur form.  Suppose that
\[
  AV = V \Lambda
\]
and let $V = UR$ be a complex QR decomposition of $V$.  Then
\[
  AU = U (R \Lambda R^{-1}) = UT
\]
is a complex Schur form for $A$.

\subsection{Symmetric eigenvalue problems and SVDs}

Broadly speaking, I tend to distinguish between two related
perspectives on eigenvalues.  The first is the linear map
perspective: $A$ represents an operator mapping a space to itself,
and an eigenvector corresponds to an {\em invariant direction}
for the operator.  The second perspective is the quadratic form
perspective: if $A$ is a symmetric matrix representing a quadratic
form $x^T A x$, then the eigenvalues and eigenvectors are the
stationary values and vectors for the {\em Rayleigh quotient}
\[
  \rho_A(x) = \frac{x^T A x}{x^T x}.
\]
If we differentiate $x^T A x - \rho_A x^T x = 0$, we have
\[
  2 \delta x^T (Ax - \rho_A x) - \delta \rho_a (x^T x) = 0
\]
which means that setting $\delta \rho_A = 0$ implies
\[
  Ax-\rho_A(x) x = 0.
\]
The largest eigenvalue is the maximum of the Rayleigh quotient,
and the smallest eigenvalue is the minimum of the Rayleigh quotient.

The singular value decomposition can be thought of as a symmetric
eigenvalue problem in several different ways.  The simplest approach
is to consider the stationary points of the function
\[
  \phi(x) = \frac{\|Ax\|^2}{\|x\|^2} = \frac{x^T A^T A x}{x^T x}.
\]
This is the (square) of the function that appears in the definition of
the operator 2-norm, and it is the Rayleigh quotient for the Gram
matrix $A^T A$.  We can also consider the functional
\[
  \psi(u,v) = \frac{u^T A v}{\|u\| \|v\|},
\]
which we can show, with some calculus and algebra,
has stationary points at solutions to the matrix eigenvalue problem
\[
\left( \begin{bmatrix} 0 & A \\ A^T & 0 \end{bmatrix} - \psi I \right)
\begin{bmatrix} u \\ v \end{bmatrix} = 0
\]
whose eigenvalues are $\{ \pm \sigma_i \}$ where $\{ \sigma_i \}$
are the singular values of $A$, 

The Rayleigh quotient is such a powerful tool that the symmetric
eigenvalue problem behaves almost like a different problem from
the nonsymmetric eigenvalue problems.  There are types of error
analysis and algorithms that work for the symmetric case and have
no real useful analogue in the nonsymmetric case.

\subsection{Power method and related iterations}

\paragraph{Power method}
The simplest iteration for computing eigenvalues is the {\em power
  method}
\begin{align*}
  \tilde{x}_{k+1} &= A x_k \\
  x_{k+1} &= \tilde{x}_{k+1}/\|\tilde{x}_{k+1}\|
\end{align*}
The iterates actually satisfy
\[
  x_{k} = \frac{A^k x_0}{\|A^k x_0\|}.
\]
and if $A = V \Lambda V^{-1}$ is an eigendecomposition, then
\[
  A^k = V \Lambda^k V^{-1} = \lambda_1^k (V D^k V^{-1})
\]
where
$D = \operatorname{diag}(1, \lambda_2/\lambda_1, \ldots, \lambda_n/\lambda_1)$.
Assuming $|\lambda_1| > |\lambda_2| \geq |\lambda_3| \geq \ldots \geq
|\lambda_n|$, we have
\[
  A^k x_0 \propto v_1 + O\left( |\lambda_2|^k/|\lambda_1|^k \right),
\]
assuming $x_0$ has some component in the $v_1$ direction when
expressed in the eigenbasis.  Hence, the power iteration converges
to the eigenvector associated with the largest eigenvalue, and the
rate is determined by the ratio of the magnitudes of the largest two
eigenvalues

\paragraph{Inverse iteration}
The problem with power iteration is that it only gives us the
eigenvector associated with the dominant eigenvalue, the one farthest
from the origin.  What if we want the eigenvector associated with the
eigenvalue nearest the origin?  A natural strategy then is {\em
  inverse iteration}:
\begin{align*}
  \tilde{x}_{k+1} &= A^{-1} x_k \\
  x_{k+1} &= \tilde{x}_{k+1}/\|\tilde{x}_{k+1}\|
\end{align*}
Inverse iteration is simply power iteration on the inverse matrix,
which has the eigendecomposition $A^{-1} = V \Lambda^{-1} V^{-1}$.
Hence, inverse iteration converges to the eigenvector associated with
the eigenvalue nearest zero, and the rate of convergence is determined
by the ratio of magnitudes of that eigenvalue and the second-furthest-away.

\paragraph{Shifts}
If we want to find an eigenvalue close to some given target value
(and not just zero), a natural strategy is {\em shift-invert}:
\begin{align*}
  \tilde{x}_{k+1} &= (A-\sigma I)^{-1} x_k \\
  x_{k+1} &= \tilde{x}_{k+1}/\|\tilde{x}_{k+1}\|
\end{align*}
The eigendecomposition of $(A-\sigma I)^{-1}$ is
$V (\Lambda - \sigma I)^{-1} V^{-1}$, and the eigenvalues
nearest $\sigma$ correspond to the
largest magnitudes for $(\lambda-\sigma)^{-1}$

\paragraph{Rayleigh quotient iteration}
A static shift-invert strategy will converge geometrically (unless the
shift is an eigenvalue, in which case convergence is instantaneous).
We can accelerate convergence by using increasingly accurate estimates
for the eigenvalue as shifts.  A natural way to estimate the
eigenvalue is using the Rayleigh quotient, which gives us the iteration
\begin{align*}
  \tilde{x}_{k+1} &= (A-\rho_A(x_k) I)^{-1} x_k \\
  x_{k+1} &= \tilde{x}_{k+1}/\|\tilde{x}_{k+1}\|
\end{align*}
Rayleigh quotient iteration converges superlinearly to isolated
eigenvalues -- quadratically in the nonsymmetric case, cubically in
the symmetric case.  Unlike static shift-invert, though, the Rayleigh
quotient iteration requires a factorization of a new shifted system
at each step.

\paragraph{Subspace iteration}
So far, we have talked only about iterations for single vectors.
{\em Subspace iteration} generalizes the power iteration idea to
multiple vectors.  The subspace iteration looks like
\[
  Q_{k+1} R_{k+1} = A Q_{k};
\]
that is, at each step we multiply an orthonormal basis for a subspace
by $A$, then re-orthonormalize using a QR decomposition.  Subspace
decomposition converges like $O(|\lambda_{m+1}|^k/|\lambda_{m}|^k)$,
where $m$ is the dimension of the subspace and the eigenvalues are
ordered in descending order of magnitude.  The difference between the
iterates and the ``true'' subspace has to be measured in terms of
angles rather than vector differences.

The tricks for accelerating power iteration -- shift-invert
and adaptive shifting -- can be applied to subspace iteration as
well as to single vector iterations.

\subsection{QR iteration}

The QR iteration is the workhorse algorithm for solving nonsymmetric
dense eigenvalue problems.  Named one of the top ten algorithms of the
20th century, the modern QR iteration involves a beautiful combination
of elementary ideas put together in a clever way.  You are {\em not}
responsible for recalling the details, but you {\em should} remember
at least two ingredients: subspace iteration and Hessenberg reduction.

\paragraph{Nesting in subspace iteration}
One of the remarkable properties of subspace iteration is that it
nests: inside the subspace iteration for a subspace of dimension $m$
sits subspace iteration for subspaces of dimension $m-1, m-2, \ldots, 1$.
Hence if $A$ has eigenvalues with distinct moduli, then the iteration
\[
  Q_{k+1} R_{k+1} = A Q_k, \quad Q_k \in \bbR^{n \times n}
\]
will produce $Q_k \rightarrow Q$ where $Q$ is the orthogonal Schur
factor for $A$.  Of course, we again need to be careful to measure
convergence by angles between corresponding columns of $Q$ and $Q_k$
rather than by the vectors themselves.  If the eigenvalues do not
have distinct moduli, then $Q$ will correspond to a set of vectors
that span nested invariant subspaces.

The first column of the $Q_k$ matrix follows an ordinary power iteration:
\[
  Q_{k+1} R_{k+1} e_1 = (Q_{k+1} e_1) r_{k+1,11} = A (Q_k e_1),
\]
and the last column of $Q_{k+1}$ follows an {\em inverse} iteration
with $A^T$:
\[
  R_{k+1} Q_k^T = Q_{k+1}^T A \implies
  (Q_k e_n)^T \propto (Q_{k+1} e_n^T) A \implies
  (Q_{k+1} e_n) \propto A^{-T} (Q_k e_n).
\]
Hence, a step of {\em shifted} subspace iteration
\[
  Q_{k+1} R_{k+1} = (A-\sigma I) Q_k
\]
effectively takes a step with a shift-invert transformation for
the last vector.

\paragraph{QR iteration}
Subspace iteration puts the emphasis on the vectors.  What about the
triangular factor $T$?  If $Q_k \in \bbR^{n \times n}$ is an
approximation for the orthogonal Schur factor, an approximation for
the triangular Schur factor is $A^{(k)}$ given by
\[
  A^{(k)} = Q_k^T A Q_k.
\]
You may recognize this as a generalization of the Rayleigh quotient.
The subspace iteration recurrence is $A Q_k = Q_{k+1} R_{k+1}$, so
\[
  A^{(k)} = Q_k^T Q_{k+1} R_{k+1} = \tilde{Q}_{k+1} R_{k+1}, \quad
  \mbox{ where } \tilde{Q}_{k+1} \equiv Q_k^T Q_{k+1}.
\]
Now, magic: we compute $A^{(k+1)}$ from
the QR factorization $A^{(k)} = \tilde{Q}_k R_k$:
\[
  A^{(k+1)} = Q_{k+1}^T A Q_{k+1}
    = \tilde{Q}_{k+1}^T A^{(k)} \tilde{Q}_{k+1}
    = R_{k+1} \tilde{Q}_{k+1}.
\]
This leads to the simplest version of the {\em QR iteration}:
\begin{align*}
  A^{(0)} &= A \\
  Q_{k+1} R_{k+1} &= A^{(k)} \\
  A^{(k+1)} &= R_{k+1} Q_{k+1}
\end{align*}

\paragraph{Shifts in QR}
The simple QR iteration only converges to the quasi-triangular
real Schur factor if all the eigenvalues have different magnitudes.
Moreover, as with subspace iteration, the rate of convergence is
limited by how close together the magnitudes of the different
eigenvalues are.  To get fast convergence, we need to include
{\em shifts}:
\begin{align*}
  A^{(0)} &= A \\
  Q_{k+1} R_{k+1} &= A^{(k)} - \sigma_k I\\
  A^{(k+1)} &= R_{k+1} Q_{k+1} + \sigma_k I
\end{align*}
Using the connection to subspace iteration, choosing
$\sigma_k = A^{(k+1)}_{nn}$ ends up being equivalent to a step of Rayleigh
quotient iteration.

\paragraph{Hessenberg reduction}
Incorporating shifts (and choosing the shifts in a clever way) is one
of two tricks needed to make QR iteration efficient.  The other trick
is to convert $A$ to {\em upper Hessenberg} form before running the
iteration, i.e. factoring
\[
  A = Q H Q^T
\]
where $Q$ is orthogonal and $H$ is zero below the first subdiagonal.
QR factorization of a Hessenberg matrix takes $O(n^2)$ time, and
running one step of QR factorization maps a Hessenberg matrix to a
Hessenberg matrix.

\subsection{Problems}

\begin{enumerate}
\item
  The {\em spectral radius} of a matrix $A$ is the maximum modulus
  of any of its eigenvalues.  Show that $\rho(A) \leq \|A\|$ for
  any operator norm.
\item
  Suppose $A \in \bbR^{n \times n}$ is a symmetric matrix an
  $V \in \bbR^{n \times n}$ is invertible.  Show that $A$ is positive
  definite, negative definite, or indefinite iff $V^T A V$ is positive
  definite, negative definite, or indefinite.
\item
  Write a Julia fragment to take {\tt numiter} steps of shift-invert iteration
  with a given shift.  You should make sure that the cost per
  iteration is $O(n^2)$, not $O(n^3)$.
\item
  Suppose $T$ is a block upper-triangular matrix with diagonal blocks
  in $\bbR^{1 \times 1}$ or $\bbR^{2 \times 2}$.  Show that the
  eigenvalues of $T$ are the diagonal values in the $1 \times 1$
  blocks together with the eigenvalue pairs from the $2 \times 2$
  blocks.
\item
  If $A U = U T$ is a complex Schur form, argue that
  $A^{-1} U = U T^{-1}$ is the corresponding complex Schur form for $A^{-1}$.
\item
  Suppose $Q_k$ is the $k$th step of a subspace iteration, and $Q_*$
  is an orthonormal basis for the subspace to which the iteration is
  converging.  Let $\theta$ be the biggest angle between a vector in
  the range of $Q_*$ and the best approximation by a vector in the
  range of $Q_k$, and show that $\cos(\theta)$ is the smallest singular
  value of $Q_k^T Q_*$.
\item
  Show that the power method for the Cayley transform matrix $(\sigma
  I + A) (\sigma I - A)^{-1}$ for $\sigma > 0$ will first converge to
  an eigenvalue of $A$ with positive real part, assuming such an
  eigenvalue exists and the iteration converges at all.
\item
  In control theory, one often wants to plot a {\em transfer function}
  \[
    h(s) = c^T (A-sI)^{-1} b.
  \]
  The transfer function can be computed in $O(n^2)$ time using a
  Hessenberg reduction on $A$.  Describe how.
\end{enumerate}

\newpage
\section{Stationary iterations}

{\em Stationary iterations} for solving linear systems are rarely used
in isolation (except for particularly nicely structured problems).
However, they are often used as preconditioners for Krylov subspace
methods.

\subsection{The splitting picture}

Let $A = M-K$ be a {\em splitting} of the matrix $A$, and consider
the iteration
\[
  M x_{k+1} = K x_k + b.
\]
The fixed point equation for this iteration is
\[
  Mx = Kx + b,
\]
which is equivalent to $Ax = b$.  Using our usual trick of subtracting
the fixed point equation from the iteration equation to get an
equation for the errors $e_k = x_k-x$, we have
\[
  M e_{k+1} = K e_k \implies e_{k+1} = R e_k, \quad R \equiv M^{-1} K.
\]
The matrix $R$ is sometimes called the {\em iteration matrix}.  The
iteration converges iff the spectral radius $\rho(R)$ is less than one;
recall that the spectral radius is the maximum of the eigenvalue
magnitudes of $R$.  A sufficient condition for convergence is that
some operator norm of $R$ is less than one, and this is often easier
to establish than a bound on the spectral radius.

Ideally, a splitting should have two properties:
\begin{enumerate}
\item It should give a convergent method.
\item Applying $M^{-1}$ should be easy.
\end{enumerate}
Some standard choices of splitting are taking $M$ to be the diagonal
of $A$ (Jacobi iteration), taking $M$ to be the upper or lower
triangle of $A$ (Gauss-Seidel iteration), or taking $M$ to be
the identity (Richardson iteration).

\subsection{The sweeping picture}

For {\em analysis}, the splitting picture is the ``right'' way to
think about stationary iterations.  In implementations, though, one often
thinks not about splitting, but about {\em sweeping}.  For example,
consider the model tridiagonal system $Tu = h^2 b$ where $T$ is a
tridiagonal matrix with $2$ on the diagonal and $-1$ on the first
super- and subdiagonal.  Written componentwise, this is
\[
  -u_{i-1} + 2 u_i - u_{i+1} = h^2 b_i, \mbox{ for } 1 \leq i \leq N
\]
with $u_0 = u_{N+1} = 0$.  A sweep operation takes each equation in
turn and uses it to solve for one of the unknowns.  For example,
a Jacobi sweep looks like
\begin{lstlisting}
  # Jacobi sweep in Julia (U is u_0 through u_{N+1}, u_0 = u_{N+1} = 0)
  for i = 1:N
    Unew[i+1] = ( h^2 * b[i] + U[i] + U[i+2] )/2
  end
  U[:] = Unew
\end{lstlisting}
while a Gauss-Seidel sweep looks like
\begin{lstlisting}
  # G-S sweep in Julia (U is u_0 through u_{N+1}, u_0 = u_{N+1} = 0)
  for i = 1:N
    U[i+1] = ( h^2 * b[i] + U[i] + U[i+2] )/2;
  end
\end{lstlisting}
This formulation is equivalent to the splitting picture, but is
arguably more natural, at least in the context of PDE discretizations.

\subsection{Convergence examples}

We usually need some structural characteristic to guarantee
convergence of a stationary iteration.  We gave two examples
in class.

\paragraph{Jacobi and diagonal dominance}
Suppose $A$ is strictly row diagonally dominant, i.e.
\[
  |a_{ii}| > \sum_{j \neq i} |a_{ij}|
\]
Then Jacobi iteration converges for linear systems involving $A$.
The proof is simply that the iteration matrix $R = M^{-1} K$ by
design has $\|R\|_\infty < 1$.

\paragraph{Gauss-Seidel and SPD problems}
Suppose $A$ is symmetric and positive definite.  Then Gauss-Seidel
iteration converges for linear systems involving $A$.  To prove this,
note that each step in a Gauss-Seidel sweep is equivalent to updating
$x_i$ (holding all other entries fixed) to minimize the {\em energy}
function
\[
  \phi(x) = \frac{1}{2} x^T A x - x^T b.
\]
If $x_*$ is the minimum energy point, then
\[
  \phi(x)-\phi(x_*) = \frac{1}{2} (x-x^*)^T A (x-x^*) = \frac{1}{2} \|x-x_*\|_A^2.
\]
If $x^{(k)} \neq x_*$, then we can show that {\em some} step moving
from $x^{(k)}$ to $x^{(k+1)}$ must reduce the energy, i.e.
\[
  \|e_{k+1}\|_A < \|e_k\|_A.
\]
This is true regardless of the choice of $x^{(k)}$.  Maximizing
over all possible $x^{(k)}$ gives us that $\|R\|_A < 1$.

\subsection{Problems}

\begin{enumerate}
\item
  Consider using Richardson iteration to solve the problem
  $(I-K) x = b$ where $\|K\| < 1$ (i.e. $M = I$).  If $x_0 = 0$,
  show that $x_k$ corresponds to taking $k$ terms in a truncated
  geometric series (a.k.a~a Neumann series) for $(I-K)^{-1}$.
\item
  If $A$ is strictly {\em column} diagonally dominant, Jacobi
  iteration still converges.  Why?
\item
  Show that if $A$ is symmetric and positive definite and
  $x_*$ is a minimizer for the energy
  \[
    \phi(x) = \frac{1}{2} x^T A x - x^T b
  \]
  then
  \[
    \phi(x)-\phi(x_*) = \frac{1}{2} (x-x_*)^T A (x-x_*).
  \]
\item
  The largest eigenvalue of the tridiagonal matrix
  $T \in \bbR^{n \times n}$ is $2 - O(n^{-2})$.
  Argue that the iteration matrix for Jacobi iteration therefore
  has spectral radius $\rho(R) = 1-O(n^{-2})$, and therefore
  \[
    \log \rho(R) = -O(n^{-2})
  \]
  Using this fact, argue that it takes $O(n^2)$ Jacobi iterations
  to reduce the error by a constant factor for this problem.
\end{enumerate}

\newpage
\section{Krylov subspace methods}

The $m$-dimensional {\em Krylov subspace} generated by $A$ and $b$ is
\[
  \calK_m(A,b)
    = \operatorname{span}\{ b, Ab, \ldots, A^{m-1} b \}
    = \{ p(A) b : p \in \calP_{m-1} \}.
\]
Krylov subspaces are phenomenally useful for two reasons:
\begin{enumerate}
\item
  All you need to explore a Krylov subspace is a subroutine
  that computes matrix-vector products.
\item
  An appropriately chosen Krylov subspace often contains good
  approximations to things we would like to compute (e.g.~eigenvectors
  or solutions to linear systems).  Moreover, we can use the
  connection to polynomials to reason about the quality of that space.
\end{enumerate}

\subsection{Arnoldi and Lanczos}

While a Krylov subspace $\calK_m(A,b)$ may be an attractive space,
the power basis $b, Ab, \ldots, A^{m-1} b$ is not an attractive basis
for that space.  Because the power basis essentially corresponds to
steps in a power iteration, successive vectors get ever closer to
the dominant eigenvector for the system.  Hence, the vectors become
increasingly linearly dependent and the basis becomes increasingly
ill-conditioned.  What we would really like is an {\em orthonormal}
basis for the nested Krylov subspaces.  We can compute such a basis
by the {\em Arnoldi process}
\begin{align*}
  q_1 &= b/\|b\| \\
  v_{k+1} &= Aq_k \\
  w_{k+1} &=
    v_{k+1} - \sum_{j=0}^{k} q_j h_{j,k+1}, & h_{j,k+1} &= q_j^T v_{k+1} \\
  q_{k+1} &=
    w_{k+1} / h_{k+1,k}, & h_{k+1,k} & = \|w_{k+1}\| 
\end{align*}
That is, to get each new vector in turn we first multiply the previous
vector by $A$, then orthogonalize.

If we write $Q_k = \begin{bmatrix} q_1 & \ldots & q_k \end{bmatrix}$,
Arnoldi computes the decomposition
\[
  A Q_k = Q_k H_k + q_{k+1} h_{k+1,k} e_k^T
\]
where $H_k$ is a $k \times k$ upper Hessenberg matrix consisting of
the coefficients that appear in the orthogonalization process.
Note that $H_k = Q_k^T A Q_k$; hence, if $A$ is symmetric, then $H_k$
is both upper Hessenerg and symmetric, i.e.~tridiagonal.  In this
case, we can compute the basis with a three-term recurrence,
orthogonalizing only against two previous vectors at each step.
The resulting algorithm is known as the {\em Lanczos} algorithm.

\subsection{Krylov subspaces for linear systems}

To solve a linear system with Krylov subspace we need two ingredients:
a Krylov subspace and a method of choosing an approximation from that
space.  The two most common approaches are:
\begin{enumerate}
\item
  Choose the approximation $\hat{x}$ that gives the smallest residual
  (in the two norm).  This is the basis of the GMRES algorithm, which
  is the default solver for nonsymmetric matrices, as well as the
  MINRES algorithm for symmetric indefinite problems.
\item
  If $A$ is symmetric and positive definite, choose $\hat{x}$ to minimize
  the energy function $\phi(x)$ over all $x$ in the space, where
  \[
    \phi(x) = \frac{1}{2} x^T A x - x^T b.
  \]
  The equation $Ax_* = b$ is exactly the equation for a stationary
  point (aka a critical point), and the only critical point of $\phi$
  is the global minimum.  The energy minimization strategy is the
  basis for the method of conjugate gradients (CG), which is the
  default Krylov subspace solver for SPD problems.
\end{enumerate}
We did not actually derive CG in lecture; even more than with the QR
iteration, the magical-looking derivation of CG tends to obscure the
fundamental simplicity of the approach.  We did briefly discuss the
properties of the error in CG, namely that the algorithm minimizes the
energy norm of the error $\|x-x_*\|_A^2$ and the inverse energy norm
of the residual, i.e. $\|Ax-b\|_{A^{-1}}^2$.

\subsection{Convergence behavior}

Let's consider the Krylov subspaces $\calK_{m}(A,b)$ and the problem
of approximating $x = A^{-1} b$ by some $\hat{x} \in \calK_m(A,b)$.
How good can the best approximation from a given Krylov subspace be?
Note that any element of $\calK_m(A,b)$ can be associated with a
polynomial of degree at most $m-1$, i.e.
\[
  \hat{x} = p(A) b, \quad p \in \calP_{m-1}.
\]
The difference between $\hat{x}$ and $x$ is
\[
  \hat{x}-x = (p(A)-A^{-1})b,
\]
or, assuming $A$ is diagonalizable,
\[
  \hat{x}-x = V (p(\Lambda)-\Lambda^{-1}) V^{-1} b.
\]
Taking norms, we have
\[
  \|\hat{x}-x\| \leq \kappa(V) \, \max_{\lambda}
  |p(\lambda)-\lambda^{-1}| \, \|b\|.
\]
That is, apart from the annoying issue of the conditioning of the
eigenvectors, we can reduce the problem of bounding the error of
the best approximation to the problem of finding a polynomial
$p(z)$ that best approximates $z^{-1}$ on the set of eigenvalues.

For the SPD case, one can bound the best-estimate behavior using a polynomial
approximation to $z^{-1}$ on an interval $[\lambda_{\min}, \lambda_{\max}]$.
The argument involves a pretty use of Chebyshev polynomials -- see,
for example, the theorem on p.~187 of the textbook -- but this bound
is often quite pessimistic in practice.  The actual convergence
depends on how clustered the eigenvalues are, and also on the nature
of the right hand side vector $b$.

\subsection{Preconditioning}

Krylov subspace methods are typically {\em preconditioned}; that is,
rather than solving
\[
  Ax = b
\]
one solves (at least notionally)
\[
  M^{-1} A x = M^{-1} b
\]
where applying $M^{-1}$ (the {\em preconditioner solve}) is assumed to
be relatively inexpensive.  Typical choices for the preconditioner
include the $M$ matrix from a stationary iteration, solves with
approximate factorizations, and methods that take advantage of
the physical meaning of $A$ (e.g.~multigrid methods).  A good
preconditioner tends to cluster the eigenvalues of $M^{-1} A$.
For CG, both the preconditioner and the matrix $A$ must be SPD.
Choosing a good preconditioner tends to be as much an art as a
science, with the best preconditioners often depending on an
understanding of the particular application at hand.

\subsection{Krylov subspaces for eigenvalue problems}

If $A Q_k = Q_k H_k + h_{k,k+1} q_{k+1}$ is an Arnoldi decomposition,
the eigenvalues of $H_k$ are often used to estimate eigenvalues of
$A$.  The columns of $Q_k$ span a Krylov subspace that may be
generated using $A$ or using some transformed matrix (e.g.
$(A-\sigma I)^{-1}$ where $\sigma$ is some shift of interest).
Hence, the Krylov subspace of $A$ contains $k$ steps of a power
iteration, possibly with a shift-invert transformation, and should
have at least the approximating power of that iteration.
Note that if $H_k v = v \lambda$, then $\hat{x} = Q_k v$ is an
approximate eigenvector with
\[
  A \hat{x} = \hat{x} \lambda + h_{k,k+1} q_{k+1} e_k^T v,
\]
i.e.
\[
  \|A \hat{x}-\hat{x} \lambda\| \leq |h_{k,k+1}| |v_k|.
\]

The Julia command {\tt eigs} computes a few of the largest
eigenvalues, smallest eigenvalues, or eigenvalues near some shift
via Arnoldi (or Lanczos in the case of symmetric problems).

\subsection{Problems}

\begin{enumerate}
\item
  Suppose $A$ is symmetric positive definite and
  $\phi(x) = x^T A x/2 - x^T b$.  Show that over all
  approximations of the form $\hat{x} = Uz$, the one
  that minimizes $\phi$ satisfies $(U^T A U) z = U^T b$.
\item
  Suppose $A$ is SPD and $\phi$ is defined as in the previous
  question.  If $\hat{x} = Uz$ minimizes the energy of
  $\phi(\hat{x})$, show that $\|Uz-\hat{x}\|_A^2$ is also minimal.
\item
  Suppose $A$ is nonsingular and has $k$ distinct eigenvalues.  Argue that
  $\calK_k(A,b)$ contains $A^{-1} b$.
\item
  Argue that the residual after $k$ steps of GMRES with a Jacobi
  preconditioner is no larger than the residual after $k$ steps of
  Jacobi iteration.
\item
  If $A$ is symmetric, the largest eigenvalue is the
  maximum value of the Rayleigh quotient $\rho_A(x)$.
  Show that computing the largest eigenvalue of $\rho_{T}(z)$
  where $T = Q^T A Q$ is equivalent to maximizing $\rho_A(x)$
  over $x$ s.t. $x = Qz$.  The largest eigenvalue of $T$
  is always less than or equal to the largest eigenvalue of $A$; why?
\end{enumerate}

\newpage
\section{Iterations in 1D}

We started the class with a discussion of equation solving
in one variable.  The goal was to get you accustomed to thinking
about certain ideas (fixed point iteration, Newton iteration)
in a less complicated setting before moving on to the more general
setting of systems of equations.

\subsection{Fixed point iteration and convergence}

\paragraph{Fixed point iteration}
A {\em fixed point} of a function $g : \bbR \rightarrow \bbR$ is a
solution to the equation
\[
  x = g(x).
\]
A one-dimensional {\em fixed point iteration} is an iteration of the
form
\[
  x_{k+1} = g(x_k).
\]

\paragraph{Convergence analysis}
Our standard recipe for analyzing the convergence of a fixed point
iteration is
\begin{enumerate}
\item Subtract the fixed point equation from the iteration equation
  to get an iteration for the error.
\item Linearize the error iteration to obtain a tractable problem
  that describes the behavior of the error for starting points
  ``close enough'' to the initial point.
\end{enumerate}
More concretely, if we write the error at step $k$ as $e_k = x_k-x_*$,
then subtracting the fixed point equation from the iteration equation
gives
\[
  e_{k+1} = g(x_* + e_k) - g(x_*).
\]
Assuming $g$ is differentiable, we have
\[
  e_{k+1} = g'(x_*) e_k + O(|e_k|^2).
\]
If $|g'(x_*)| < 1$, then the fixed point is {\em attractive}: 
that is, the iteration will converge to $x_*$ for starting points
close enough to $x_*$.

\paragraph{Plotting convergence}
When $g$ is differentiable and $0 < |g'(x_*)| < 1$, fixed point
iteration is {\em linearly convergent}.  That is, we have
\[
  |e_k| \approx |e_0| |g'(x_*)|^k,
\]
and so when we plot the error on a semi-logarithmic scale, we see
\[
  \log |e_k| \approx k \log |g'(x_*)| + \log |e_0|,
\]
i.e. the (log scale) errors fall approximately on a straight line.
Of course, this convergence behavior only holds until rounding error
starts to dominate!  When $g'(x) = 0$, we have {\em superlinear}
convergence.

\subsection{Newton's method}

\paragraph{Newton's method}
Newton's method for solving $f(x) = 0$ is:
\[
  x_{k+1} = x_k - f'(x_k)^{-1} f(x_k).
\]
We derive the method by taking the linear approximation
\[
  f(x_k + p) \approx f(x_k) + f'(x_k) p
\]
and choosing the update $p$ such that the approximation is zero;
that is, we solve the linear equation
\[
  f(x_k) + f'(x_k) (x_{k+1}-x_k) = 0.
  \]

\paragraph{Local convergence}
Assuming $f$ is twice differentiable, the {\em true} solution $x_*$ satisfies
\[
  f(x_k) + f'(x_k) (x_*-x_k) = O(|x_k-x_*|^2)
\]
Writing $e_k = x_k-x_*$ and subtracting the true solution equation
from the iteration equation gives us
\[
  f'(x_k) e_{k+1} = O(|e_k|^2).
  \]
If $f'(x_k)$ is bounded away from zero, we then have that $|e_{k+1}| =
O(|e_k|^2)$, or {\em quadratic convergence}.  Plotting quadratic
convergence on a semi-logarithmic plot gives us a shape that looks
like a downward-facing parabola, up to the point where roundoff errors
begin to dominate (which often only takes a few steps).

\paragraph{Initial guesses}
Newton's iteration is {\em locally convergent} -- it is only
guaranteed to converge from starting points that are sufficiently
close to the solution.  Hence, a good initial guess can be critically
important.  Getting a good initial guess frequently involves reasoning
about the problem in some application-specific way, approximating the
original equations in a way that yields something analytically
tractable.

\paragraph{Secant iteration}
One of the annoying features of Newton's iteration is that
it requires that we compute derivatives.  Of course, we
can always replace the derivatives by a {\em finite difference}
approximation:
\[
  f'(x_{k}) \approx \frac{f(x_k)-f(x_{k-1})}{x_k-x_{k-1}}.
\]
This leads to the {\em secant iteration}
\[
  x_{k+1} = x_k - \frac{f(x_k)(x_k-x_{k-1})}{f(x_k)-f(x_{k-1})}
\]
The convergence analysis for secant iteration is slightly more
complicated than that for Newton iteration, but the iteration
is certainly superlinear.

\subsection{Bisection}

Newton's iteration -- and most other fixed point iterations --
generally only converge if the initial guess is good enough.  An
alternate approach of {\em bisection} converges slowly but
consistently to a solution of $f(x) = 0$ in an interval $[a,b]$
assuming that $f(a) f(b) < 0$ and $f$ is continuous on $[a,b]$.
Bisection relies on the idea that if $f$ changes sign between
the endpoints of an interval, then there must be a zero somewhere
in the interval.  If there is a sign change between $f(a)$ and $f(b)$
and $c = (a+b)/2$, then there are three possibilities:
\begin{itemize}
\item $f(c) = 0$ (in which case we're done).
\item $f(c)$ has the same sign as $f(a)$, in which case $[c,b]$
  contains a zero of $f$.
\item $f(c)$ has the same sign as $f(b)$, in which case $[a,c]$
  contains a zero of $f$.
\end{itemize}
Thus, we have an interval half the size of $[a,b]$ that again
contains a solution to the problem.

Bisection produces a sequence of ever-smaller intervals, each
guaranteed to contain a solution.  If we know there is a solution
in the interval $[a,b]$, we usually take $x = (a+b)/2$ as the
approximation; barring any additional information about the solution,
this is the approximation in the interval that minimizes the
worst-case error.  Hence, if $[a,b]$ is the initial interval and
$x_0 = (a+b)/2$ is the initial guess, then the initial error bound is
$|x_0-x_*| \leq |b-a|/2$.  For successive iterations, the error bound
is $|x_k-x_*| \leq |b-a|/2^{k+1}$.

\subsection{Combined strategies}

Newton and secant iterations are fast but dangerous.  Bisection is
slow but steady.  We would like the best of both worlds%
\footnote{Does this sound like a blurb for a bad romance novel?}:
superlinear convergence close to the solution, with steady progress
even far from the solution.  {\em Brent's algorithm} is one example
that does this.  Of course, Brent's algorithm
still requires an initial bracketing interval, but it is otherwise
about as bulletproof as these things can possibly be.

\subsection{Sensitivity analysis}

Suppose $\hat{x}$ is an approximation of $x_*$ such that $f(x_*) = 0$,
where $f$ is at least continuously differentiable.  How can we
evaluate the quality of the estimate?  The simplest thing is to check
the {\em residual} error $|f(\hat{x})|$.  In some cases, this is
enough -- we really care about making $|f|$ small, and any point that
satisfies this goal will suffice.  In other cases, though,
we care about the {\em forward} error $|\hat{x}-x_*|$.  Of course,
if we have an estimate of the derivative $f'(x_*)$, then we can
use a Taylor expansion to estimate
\[
  |\hat{x}-x_*| \approx |f(\hat{x})|/|f'(x_*)|
                \approx |f(\hat{x})|/|f'(\hat{x})|.
\]
You should recognize this as saying that the Newton correction
starting from $\hat{x}$ is a good estimate of the error.  Of course,
if we are able to compute both $f(\hat{x})$ and $f'(\hat{x})$
accurately, it may make sense to compute a Newton update directly!
One of the standard termination criteria for Newton iteration involves
using the size of the last correction as a (very conservative)
estimate of the error at the current step.

There are two caveats here.  First, there is often some rounding error
in our computation of $f$, and we may need to take this into account.
Assuming we can compute $|f'(\hat{x})|$ reasonably accurately, and
$|\hat{f}(\hat{x})-f(\hat{x})| < \delta$ where $\hat{f}$ is the value
of $f(\hat{x})$ computed with roundoff, then we have
\[
  |\hat{x}-x_*| \lesssim (|\hat{f}(\hat{x})|+\delta)/|f'(\hat{x})|.
\]
Thus, $\delta/|f'(\hat(x)|$ estimates the best error we could
reasonably expect.

The second caveat is that sometimes $f'(x_*) = 0$, or is incredibly
close to zero.  In this case, we can still pursue the same type of
analysis, but we need to take additional terms in the Taylor
expansion.  Or, if we know in advance that $f'(x_*) = 0$, we may
choose to find a root of $f'$ rather than finding a root of $f$.

\subsection{Problems}

\begin{figure}
  \begin{tikzpicture}
    \begin{semilogyaxis}[
        width=\textwidth,
        height=4cm,
        grid={major}
      ]
      \addplot table[x=k,y=N] {fig/rev_itercomp.dat};
      \addplot table[x=k,y=G] {fig/rev_itercomp.dat};
      \addplot table[x=k,y=B] {fig/rev_itercomp.dat};      
    \end{semilogyaxis}
  \end{tikzpicture}
  \caption{Convergence of Newton iteration, a fixed point iteration,
    and bisection for $\cos(x) = 0$.}
  \label{fig:itercomp}
\end{figure}

\begin{enumerate}
\item
  Consider the fixed point iteration $x_{k+1} = g(x_k)$ and assume
  $x_*$ is an attractive point.  Also assume $|g''(x)| < M$
  everywhere.  We know that the iteration converges to $x_*$ from
  ``close enough'' starting points; show that a sufficient condition
  for convergence is
  \[
    |x_0-x_*| < \frac{2(1-g'(x_*))}{M}.
  \]
\item
  What is Newton's iteration for finding $\sqrt{a}$?
\item
  Consider the fixed-point iteration $x_{k+1} = x_k + \cos(x_k)$.
  Show that for $x_0$ near enough to $x_* = \pi/2$, the iteration
  converges, and describe the convergence behavior.
\item
  The graphs shown in Figure~\ref{fig:itercomp} show the convergence
  of Newton's iteration starting from $x_0 = 1$, the fixed point iteration
  $x_{k+1} = x_k + \cos(x_k)/x_k$ starting from $x_0 = 1$
  and bisection starting from $[0,2]$ to the solution of
  $\cos(x) = 0$.  Which plot corresponds to which method?  How can you
  tell?
\item
  Find an example of a function with a unique zero and a starting
  value such that Newton's iteration does not converge.
\item
  Suppose $f$ has a sign change for between $a = 1000$ and $b = 1001$.
  How many steps of bisection are required to obtain a {\em relative}
  error of $10^{-6}$?
\end{enumerate}

\newpage
\section{Multivariate nonlinear problems}

In the last part of the class, we moved from problems in numerical
linear algebra (simple for one reason) and nonlinear equations in one
variable (simple for another reason) to problems involving nonlinear
equation solving and optimization with many variables.  The picture is
similar to the one we saw in 1D, but more complicated both due to the
fact that we're now dealing with several dimension and due to the fact
that our safe fallback method in 1D (bisection) does not generalize
nicely to higher-dimensional problems.

\subsection{Nonlinear equations and optimization}

We are interested in two basic problems:
\begin{enumerate}
\item Given $F : \bbR^n \rightarrow \bbR^n$ a twice-differentiable
  function, solve $F(x_*) = 0$.
\item Find a local minimum of $\phi : \bbR^n \rightarrow \bbR$ a
  differentiable function with three derivatives.
\end{enumerate}
The two problems are not so far apart: finding a zero of $F$ is
equivalent to minimizing $\|F\|^2$, while a local minimum of
$\phi$ occurs at a stationary point, i.e.~$x_*$ satisfying the
nonlinear equation $\nabla \phi(x_*) = 0$.  The optimization
perspective is particularly useful for analyzing ``globalized''
iterations (trust region methods and line search methods).

\subsection{Fixed point iterations}

As in one space dimension, our basic tool is a fixed point iteration:
\[
  x_{k+1} = G(x_k)
\]
converging to a stationary point
\[
  x_* = G(x_*).
\]
Letting $e_k = x_k-x_*$, we have
\[
  e_{k+1} = G'(x_*) e_k + O(\|e_k\|^2),
\]
and so we have convergence for small enough $e_0$ when
$\rho(G'(x_*)) < 1$.  If $G'(x_*) = 0$, we have superlinear
convergence.

\subsection{Newton's method for systems}

Newton's method for solving nonlinear systems is
$x^{(k+1)} = x^{(k)} + p^{(k)}$ where
\[
  F(x^{(k)}) + F'(x^{(k)}) p^{(k)} = 0.
\]
Put differently,
\[
  x^{(k+1)} = x^{(k)} - F'(x^{(k)})^{-1} F(x^{(k)}).
\]
The iteration is quadratically convergent if $F'(x^*)$ is
nonsingular.

As an example, consider the problem of finding the intersection
between the unit circle and the unit hyperbola, i.e., finding a
zero of
\[
  F(x,y) = \begin{bmatrix} x^2 + y^2 - 1 \\ xy - 1 \end{bmatrix}.
\]
The Jacobian of $F$ is
\[
  F' =
  \begin{bmatrix}
    \frac{\partial F_1}{\partial x} &
    \frac{\partial F_1}{\partial y} \\
    \frac{\partial F_2}{\partial x} &
    \frac{\partial F_2}{\partial y}
  \end{bmatrix} =
  \begin{bmatrix}
    2x & 2y \\
    y & x
  \end{bmatrix}
\]
Note that the Jacobian is singular when $x = y$; that is, not only is
the Newton iteration only locally convergent, but the Newton step may
be impossible to solve at some points.

\subsection{Newton's method for optimization}

For optimization problems, Newton's method involves finding a
stationary point, i.e.~a point at which the gradient $\nabla \phi$ is
zero.  However, a stationary point could also be a local maximum or
a saddle point, so for optimization we typically only use Newton
steps if we can guarantee that they will decrease the objective
function.  Otherwise we modify the Newton iteration to guarantee
that we choose a {\em descent direction} for our step.

The Jacobian of the gradient is the matrix of second derivatives
$H_{\phi}$, also known as the Hessian matrix.  At a local minimum, the
Hessian must at least be positive semidefinite; at and near a {\em
  strong} local minimum, the Hessian must be positive definite.  A
pure Newton step for optimization is
\[
  p_k = -H_{\phi}(x_k)^{-1} \nabla \phi(x_k)
\]
We can think of this as running Newton on the gradient system
or as finding a stationary point of the local quadratic
approximation
\[
  \phi(x_k + p) \approx
  \phi(x_k) + \nabla \phi(x_k)^T p + \frac{1}{2} p^T  H_{\phi}(x_k)^{-1} p.
\]

We would like to guarantee that steps move ``downhill'', i.e.
that $p_k$ is a descent direction:
\[
  \nabla \phi(x_k)^T p_k < 0.
\]
When $H_{\phi}(x_k)$ is positive definite, so is $H_{\phi}(x_k)^{-1}$,
and so in this case $\nabla \phi(x_k)^T p_k = -\nabla \phi(x_k)^T p_k
H_{\phi}(x_k)^{-1} \nabla \phi(x_k) < 0$ and we do have a descent
direction.  When $H_{\phi}$ is indefinite, we typically modify our
step to guarantee a descent direction.  That is, we consider an
iteration with steps
\[
  p_k = -H_k^{-1} \nabla \phi(x_k)
\]
where $H_k$ is a positive definite scaling matrix, chosen to be the
Hessian when that is positive definite and something close to the
Hessian (e.g. $H_k = H_{\phi}(x_k) + \eta I$, or something based on
a modified factorization of $H_{\phi}(x_k)$).

\subsection{Gauss-Newton and nonlinear least squares}

The {\em nonlinear least squares} problem is to minimize
\[
  \phi(x) = \frac{1}{2} \|F(x)\|^2 = \frac{1}{2} F(x)^T F(x),
  \quad F : \bbR^{n} \rightarrow \bbR^{m}.
\]
The gradient of $\phi$ is
\[
  \nabla \phi(x) = J(x)^T F(x), \quad J(x) = F'(x),
\]
and the Hessian of $\phi$ is the matrix with entries
\[
H_{\phi,ij} = (J^T J)_{ij} +
\sum_k \frac{\partial^2 F_j}{\partial x_i \partial x_k} F_k(x).
\]
The latter term is often a pain to compute, and when the least squares
problem can be solved so that $F$ has a small residual (i.e. $F(x_*)
\approx 0$), we might want to discard it.  This leads us to the
{\em Gauss-Newton} iteration
\[
  x_{k+1} = x_k + p_k, \quad p_k = -(J^T J)^{-1} (J^T F) = -J^\dagger F.
\]
Alternately, we can think of the Gauss-Newton iteration as minimizing
the linearized residual approximation
\[
  F(x_k + p) \approx F(x_k) + J(x_k) p.
\]

When $n = m$ and the Jacobian is nonsingular, Gauss-Newton iteration
is the same as Newton iteration on the equation $F(x) = 0$.
Otherwise, Gauss-Newton is {\em not} the same as Newton iteration for
the least squares optimization problem, and it does not converge
quadratically unless the solution satisfies $F(x_*) = 0$.  On the
other hand, the linear convergence is often more than adequate (and it
can be accelerated if needed).

\subsection{Problems with Newton}

There are a few drawbacks to pure Newton (and Gauss-Newton)
iterations.
\begin{itemize}
\item The iterations are only locally convergent.  We therefore want
  either good initial guesses (application specific) or approaches
  that {\em globalize} the iterations, expanding the region in which
  they converge.  Often, we need to use both strategies.
\item Newton iteration involves computing first and second
  derivatives.  This is fine for simple functions of modest size,
  and in principle automated differentiation tools can compute the
  relevant derivatives for us if we have access to the source code
  for a program that computes the function.  On the other hand,
  we don't always have that luxury -- someone may hand us a ``black
  box'' function for which we lack source code, for example --
  and so sometimes it is difficult to get the relevant derivatives.
  Even if computing the derivatives is not difficult, it may be
  unappealingly expensive.
\item Even when we can get all the relevant derivatives,
  Newton iteration requires factoring a new matrix (Jacobian or
  Hessian) at every step.  The linear algebra costs may again
  be expensive.
\end{itemize}
For all these reasons, Newton iteration is not the ending point
of linear solvers, but a starting point.

\subsection{Approximating Newton}

There are two main approaches to approximating Newton steps.
First, one can use an {\em inexact} Newton approach,
solving the Newton linear systems approximately.  For example,
in {\em Newton-Krylov} methods, we would apply a (preconditioned)
Krylov subspace solver to the linear systems, and we might choose
to terminate the solver while the residual for the linear system
was not completely zero.  That is, there is a tradeoff between how
many linear iteration steps we take and how many nonlinear iteration
steps we take.  We did some analysis on a homework problem
to show that for an optimization problem in which we solve the linear
system with residual $r$, if $\kappa(H) \|r\| < \|\nabla \phi\|$
then we are at least guaranteed a descent direction.

Inside a Newton-Krylov solver, one repeatedly computes matrix
vector products with the Jacobian matrix $J$.  It is not always
necessary to compute the Jacobian explicitly to form these
matrix-vector products.  Indeed, it may not even be necessary to
compute the Jacobian analytically; note that
\[
  J v = F'(x) v = \frac{\partial F}{\partial v}(x)
  = h^{-1} (F(x+hv)-F(x)) + O(h).
\]
Of course, this still leaves the question of
how to choose the finite difference step size $h$!

The second family of methods are {\em quasi} Newton methods,
in which we use an approximation of the Jacobian or Hessian.
The most popular quasi-Newton approach is the BFGS algorithm
(and the L-BFGS variant), which builds up a Hessian approximation
by updating an initial approximation with information obtained
by looking at successive iterates.  We mentioned these methods
briefly in lecture, and the book mentions them as well,
but did not go into detail.

\subsection{Other first-order methods}

In addition to inexact Newton methods and quasi-Newton methods, there
are a plethora of first-order methods that don't look particularly
Newton like, at least at first glance.  For optimization, for example,
classic steepest descent methods are usually introduced before Newton
methods (though steepest descent is often very slow).  There are also
methods related to classical stationary iterations for linear systems.
For example, the cyclic coordinate descent method for optimization
considers each variable in turn and adjusts it to reduce the objective
function value.  When applied to a quadratic objective function,
cyclic coordinate descent becomes Gauss-Seidel iteration.

\subsection{Globalization: line search}

So far, we have only discussed how to choose an update $p$
that ``looks promising'' (i.e.~is a descent direction for an
optimization problem).  This may involve an expensive subcomputation
as in Newton's method, or a simple-minded choice like $p = \pm e_j$
as in cyclic coordinate descent.  But just because an update $p_k$
looks promising based on a simplified linear or quadratic model of
the objective function does not mean that $x_{k+1} = x_k + p_k$ will
actually be better than $x_k$; indeed, the objective function at
$x_{k+1}$ (or the norm of the residual in the case of equation
solving) may be {\em worse} than it was at $x_k$.

A {\em line search} strategy uses the update
\[
  x_{k+1} = x_k + \alpha_k p_k
\]
for some $0 < \alpha_k < 1$ chosen to guarantee a reduction in the
objective function value.  A typical strategy is to start by trying
$\alpha_k = 1$, then cut $\alpha_k$ in half if the step does not
look ``good enough'' according to some criterion.  A common criterion
is the {\em Armijo} condition, which says that we need to make at
least some fixed fraction of the progress predicted by the linear
model; that is, we require
\[
  \phi(x_{k+1})-\phi(x_k) < \eta \alpha_k \nabla \phi(x_k)^T p
\]
for some constant $\eta < 1$.  A simpler criterion (and one that can
cause nonconvergence, at least in principle) is to simply insist
\[
  \phi(x_{k+1}) < \phi(x_k).
\]

With an appropriate line search strategy and a condition on the
search direction, we can obtain optimization strategies that always
converge to a local minimum, unless the objective function is
asymptotically flat or decreasing in some direction so that the
iteration can ``escape to infinity.''  The appropriate {\em Wolfe
  conditions} are spelled out in detail in optimization classes,
but we pass over them here.

\subsection{Globalization: trust regions}

In line search, we first pick a direction and then decide how far to
go in that direction to guarantee progress.  But if we proposed a
Newton step and line search told us we couldn't take it, that means
that the step fell outside the range where we could really trust the
quadratic approximation on which the step was based.  The idea of {\em
  trust region} methods is to try to choose a step that minimizes the
function within some region in which we trust the model.  If we fail
to make adequate progress, we reduce the size of our trust region; if
the model proves highly accurate, we might expand the trust region.

The trust region subproblem is to minimize a quadratic approximation
\[
  \psi(x_k+p) = \phi(x_k) + \nabla \phi(x_k)^T p + \frac{1}{2} p^T H p
\]
subject to the constraint that $\|p\| \leq \rho$ for some given
$\rho$.  At the solution to this problem, we satisfy the critical
point equation
\[
  (H+\lambda I) p = -\nabla \phi(x)
\]
for some $\lambda \geq 0$.  If the ordinary Newton step falls inside
the trust region, then $\lambda = 0$ and the constraint is said to be
{\em inactive}.  Otherwise, we choose $\lambda$ so that
$\|p\| = \rho^2$.  Alternately, we may focus on $\lambda$,
leaving the trust region radius $\rho$ implicit.

Applied to the Gauss-Newton iteration, the trust region approach
yields subproblems of the form
\[
  (J^T J + \lambda I) p = -J^T F.
\]
You may recognize this as equivalent to solving the least-squares
problem for the Gauss-Newton update with Tikhonov regularization.
This {\em Levenberg-Marquardt} update strategy actually pre-dates the
development of the trust region framework.

Because it potentially involves searching for an appropriate
$\lambda$, the trust region subproblem is more expensive than an
ordinary Newton subproblem, which may already be rather expensive.
Because of this, the trust region subproblem is usually only solved
approximately (using the colorfully-named dogleg strategy, for
example).

While trust region methods are somewhat more complicated to implement
than line search methods, they frequently are able to solve problems
with fewer function evaluations.

\subsection{Globalization: continuation methods}

Even with globalization, Newton and Newton-like iterations will not
necessarily converge quickly without a good initial guess.  Moreover,
if there are multiple solutions to a problem, Newton may converge to
the ``wrong'' (in the light of the application) solution without a
good initial guess.  Most of the time, finding a good initial guess is
a matter of manipulating application-specific approximations.
There is, however, one general-purpose strategy that often works:
continuation.

The idea of continuation methods is to study not one function, but a
parametric family.  The parameter may be a physically meaningful
quantity (e.g.~magnitude of a force, displacement, voltage, etc); or
it may be a purely artificial construct.  By gradually changing the
parameter, one can move from an easy problem instance to a hard
problem instance in a controlled way.  Continuation methods follow a
{\em predictor-corrector} pattern: given a solution at parameter value
$s_k$, one first {\em predicts} the solution at parameter value
$s_{k+1}$ and then {\em corrects} the prediction using some
locally-convergent iteration.  For example, the predictor might be the
trivial predictor (i.e.~use as an initial guess at $s_{k+1}$ the
converged solution at $s_k$) and a Newton corrector.  If the iteration
diverges, one can always try again with a shorter step size.

The other advantage provided by continuation methods is that we are
not forced to use just one parameter.  We can choose between
parameters, or even make up a new parameterization (this is the basis
for {\em pseudo-arclength} strategies, which I mentioned in a
throw-away sentence one lecture).  Often problems that seem difficult
when using one parameter are trivial with respect to a different
parameter.

\subsection{Problems}

\begin{enumerate}
\item
  Write a Julia code to estimate $\alpha$ and $x$ such that
  $y = \alpha x$ is tangent to $y = \cos(x)$ near $x_0 = n \pi$
  for $n > 0$.  I recommend writing two equations (matching function
  values and matching derivatives) in two unknowns (the intersection
  $x$ and $\alpha$) and applying Newton.  What is a good initial guess?
\item
  Write a Julia code to find a critical point of
  $\phi(x,y) = -\exp(x^2+y^2) (x^2 + y^2 - 2(ax + by) + c)$
  using Newton's iteration.
\item
  Write a Julia fragment to minimize $\sum_j \exp(r_j^2)-1$ where
  $r = Ax-b$.  Use a Gauss-Newton strategy (no need to bother
  with safeguards like line search).
\item
  Consider the fixed point iteration
  \[
    x_{k+1} = x_k - A^{-1} F(x_k)
  \]
  where $F$ has two continuous derivatives and $A$ is some (possibly
  crude) approximation to the Jacobian of $F$ at the solution $x_*$.
  Under what conditions does the iteration converge?
\item
  Suppose $x_*$ is a strong local minimum for $\phi$, i.e. $\nabla
  \phi(x_*) = 0$ and $H_\phi(x_*)$ is positive definite.  For starting
  points $x_0$ close enough to $x_*$, Newton with line search based
  on the Armijo condition behaves identically to an unguarded
  Newton iteration with no line search.  Why?
\item
  Argue that for large enough $\lambda$,
  $p = -(H_{\phi}(x) + \lambda I)^{-1} \nabla \phi(x)$ is guaranteed
  to be a descent direction, assuming $x$ is not a stationary point.
\item
  Suppose $F : \bbR^n \times \bbR \mapsto \bbR^n$ (i.e. $F = F(x,s)$).
  If we solve $F(x(s), s) = 0$ for a given $s$ using Newton's
  iteration and we are able to compute $\partial F/\partial s$
  in at most $O(n^2)$ time, we can compute $dx/ds$ in $O(n^2)$ time.
  How?
\item
  Describe a fast algorithm to solve
  \[
    Ax = b(x_n)
  \]
  where $A \in \bbR^{n \times n}$ is a fixed matrix and
  $b : \bbR \rightarrow \bbR^n$ is twice differentiable.
  Your algorithm should cost $O(n^2)$ per step and converge quadratically.
\end{enumerate}

\newpage
\section{Constrained problems}

Most of our discussion of optimization involved {\em unconstrained}
optimization, but we did spend two lectures talking about the
constrained case (and the overview section in the book is pretty
reasonable).  The constrained problem is
\[
  \mbox{minimize } \phi(x) \mbox{ s.t.~} x \in \Omega
\]
where $\Omega \subset \bbR^n$ is usually defined in terms of a
collection of equations and inequalities
\[
  \Omega = \{ x \in \bbR^n : g(x) = 0 \mbox{ and } h(x) \leq 0 \}.
\]
We discussed three basic approaches to constraints: elimination,
barriers and penalties, and Lagrange multipliers.  Each can be
used for both theory and as the basis for algorithms.

\subsection{Constraint elimination}

The idea behind {\em constraint elimination} is that a set of equality
constraints $g(x) = 0$ implicitly define a lower-dimensional surface
in $\bbR^n$, and we can write the surface as a parametric function
$x = F(y)$ for $y \in \bbR^p$ for $p < n$.
Then the constrained problem in $x$ is an unconstrained problem in $y$:
\[
  \mbox{minimize } \phi(x) \mbox{ s.t.~} g(x) = 0 \quad \equiv \quad
  \mbox{minimize } \phi(F(y)).
\]
The chief difficulty with constraint elimination is that we have to
find a parameterization of the solutions to the constraint equations.
This is hard in general, but is straightforward when the constraints
are linear: $g(x) = A^T x - b$.  In that case we can use a full (not
economy) QR decomposition of $A$ to parameterize all feasible $x$ as
\[
  x = F(y) = Q_1 R_1^{-T} b + Q_2 y.
\]
In terms of the linear algebra, linear constraint elimination has
some attractive features: if $\phi$ is convex, then so is
$\phi \circ F$; and the Hessians of $\phi \circ F$ are better
conditioned than those of $\phi$.  On the other hand, even linear
constraint elimination will generally destroy sparsity of the problem.
  
Constraint elimination is also an attractive option for some classes
of problems involving linear {\em inequality} constraints,
particularly if those constraints are simple (e.g.~elementwise
non-negativity of the solution vector).  One can either solve
the inequality-constrained problem by an iteration that incrementally
improves estimates of the active set and solves equality-constrained
subproblems.  Alternately, one might use a parameterization that
removes the need for the inequality constraint; for example, we
can parameterize $\{ x \in \bbR : x \geq 0 \}$ as $x = y^2$ where
$y$ is unconstrained.  However, while this is simple to think about,
it may not be the best approach numerically.  For example, an
objective that is convex in $x$ may no longer by convex in $y$.

Ideally, you should understand constraint elimination well enough to
implement it for linear constraints and simple objectives
(e.g.~quadratic objectives and linear least squares).

\subsection{Penalties and barriers}

The idea behind penalties and barriers is that we add a term to
the objective function to (approximately) enforce the constraint.
For penalty methods, we add a term that is positive if the constraint
is violated, and grows quickly as the level of violation becomes
worse.  For barrier methods, we add a term that is positive near the
boundary and blows up to infinity at (and outside) the boundary.
Often the penalty or barrier depends on some {\em penalty parameter} $\mu$,
and the exact solution is recovered in the limit as $\mu \rightarrow 0$.
A common example is to enforce an equality constraint by a quadratic
penalty:
\[
  x(\mu) = \mbox{argmin } \phi(x) + \frac{1}{2\mu} g(x)^2.
\]
As $\mu \rightarrow 0$, $x(\mu)$ converges to $x_*$, a constrained
minimizer of $\phi$ subject to $g(x) = 0$.

Unfortunately the conditioning of the Hessian scales like
$O(\mu^{-1})$, so the problems become increasingly numerically
sensitive with larger $\mu$.  One way of dealing with this sensitivity
is to get increasingly better initial guesses by tracing $x(\mu)$
through a sequence of ever smaller $\mu$ values.  One can do the same
thing for inequality constraints; this idea, together with a
logarithmic barrier, is the main tool in {\em interior point} methods.

In the case of {\em exact} penalties, the exact solution may be
recovered for nonzero $\mu$; but exact penalty methods often result
in a non-differentiable objective.

Ideally, you should understand the basic idea behind penalties at the
level where you could implement a simple penalty method (or barrier
method).  You should also understand enough about sensitivity and
conditioning of linear systems to understand how well a quadratic
penalty with a quadratic objective does at approximating the solution
to an equality-constrained least squares problem.

\subsection{Lagrange multipliers}

Just as solutions to unconstrained optimization problems
occur at stationary points, so do the solutions to constrained
problems; we just have to work with a slightly different functional.
To minimize $\phi(x)$ subject to $g(x) = 0$ and $h(x) \leq 0$,
we form the {\em Lagrangian}
\[
  L(x,\lambda,\mu) = \phi(x) + \lambda^T g(x) + \mu^T h(x).
\]
where $\lambda$ and $\mu$ are vectors of {\em Lagrange multipliers}.
In class, we described these multipliers by a physical analogy as
forces that enforce the constraint.
At a critical point, we have the {\em KKT} conditions:
\begin{align*}
  \frac{\partial L}{\partial x} &= 0 & \mbox{(stationarity)}\\
  \mu^T h(x) & = 0 & \mbox{(complementary slackness)} \\
  g &= 0 \mbox{ and } h \geq 0 & \mbox{(primal feasibility)} \\
  \mu &\geq 0 & \mbox{(dual feasibility)}
\end{align*}
If $\mu_i > 0$, we say the $i$th inequality constraint is {\em active}.

There are two major ways of dealing with constraints in solvers.
{\em Active set} methods guess which constraints are active,
and then solve an equality constrained sub-problem.  If the guess
was wrong, one adjusts the choice of active constraints and solves
another equality constrained sub-problem.  In contrast, methods
based on {\em penalties} or {\em barriers} deal with inequality
constraints by adding a cost to $\phi$ that gets big as one
gets close to the boundary of the feasible region, either from the
inside (barriers) or from the outside (penalties).  
{\em Interior point methods}, which are among the most popular
and successful constrained optimization solvers, use a parameterized
barrier function together with a continuation strategy to drive the
barrier parameter toward a small value that yields hard
(ill-conditioned) subproblems but with accurate results.

You should ideally understand the special case of setting and running
Newton iteration for a Lagrangian function associated with an
equality-constrained optimization problem.  Determining the
active set for the inequality-constrained case is trickier,
and probably would not be a great source of questions for
an exam.

\subsection{Problems}

\begin{enumerate}
\item
  Describe three algorithms to minimize $\|Ax-b\|^2$ subject to
  $Cx = d$, where $A \in \bbR^{m \times n}$, $m > n$ and
  $C \in \bbR^{p \times m}$, $p < m$: one based on constraint
  elimination, one a
  quadratic penalty formulation, and one via Lagrange multipliers.
\item
  Write a system of equations to characterize the minimum of
  a linear function $\phi(x) = v^T x$ on the ball $x^T M x = 1$.
\item
  Suppose $A$ is symmetric and positive definite and consider
  the quadratic objective
  \[
    \phi(x) = \frac{1}{2} x^T A x - x^T b.
  \]
  Suppose the global optimum has some negative components.
  We can find the constrained optimum be solving a sequence of
  logarithmic barrier problems
  \[
    \mbox{minimize } \phi(x) - \mu \sum_i \log(x_i).
  \]
  Write a Newton iteration to solve this problem.
\end{enumerate}

\end{document}
