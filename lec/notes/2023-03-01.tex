\documentclass[12pt, leqno]{article}
\input{common}

\begin{document}
\hdr{2023-03-01}

Recall from last time that when there are many solutions that give an
almost minimal residual $Ax-b$, we generally {\em regularize} the
least squares problem by adding some assumption about which possible
solutions are more likely or somehow more preferable.
Different statistical assumptions give
rise to different regularization strategies; for the current
discussion, we shall focus on the computational properties
of a few of the more common regularization strategies without going
into the details of the statistical assumptions.  In particular,
we consider four strategies in turn
\begin{enumerate}
\item {\em Factor selection} via {\em pivoted QR}.
\item {\em Tikhonov regularization} and its solution.
\item {\em Truncated SVD regularization}.
\item {\em $\ell^1$ regularization} or the {\em lasso}.
\end{enumerate}

\section{Factor selection and pivoted QR}

In ill-conditioned problems, the columns of $A$ are nearly linearly
dependent; we can effectively predict some columns as linear
combinations of other columns.  The goal of the column pivoted QR
algorithm is to find a set of columns that are ``as linearly
independent as possible.''  This is not such a simple task,
and so we settle for a greedy strategy: at each step, we select the
column that is least well predicted (in the sense of residual norm)
by columns already selected.  This leads to the {\em pivoted QR
  factorization}
\[
  A \Pi = Q R
\]
where $\Pi$ is a permutation and the diagonal entries of $R$ appear
in descending order (i.e. $r_{11} \geq r_{22} \geq \ldots$).  To
decide on how many factors to keep in the factorization, we either
automatically take the first $k$ or we dynamically choose to take $k$
factors where $r_{kk}$ is greater than some tolerance and
$r_{k+1,k+1}$ is not.

The pivoted QR approach has a few advantages.  It yields {\em
  parsimonious} models that predict from a subset of the columns of
$A$ -- that is, we need to measure fewer than $n$ factors to produce
an entry of $b$ in a new column.  It can also be computed relatively
cheaply, even for large matrices that may be sparse.

\section{Tikhonov}

A second approach is to say that we want a model in which the
coefficients are not too large.  To accomplish this, we add
a penalty term to the usual least squares problem:
\[
  \mbox{minimize } \|Ax-b\|^2 + \lambda^2 \|x\|^2.
\]
Equivalently, we can write
\[
\mbox{minimize } \left\|
\begin{bmatrix} A \\ \lambda I \end{bmatrix} x -
\begin{bmatrix} b \\ 0 \end{bmatrix}
\right\|^2,
\]
which leads to the regularized version of the normal equations
\[
  (A^T A + \lambda^2 I) x = A^T b.
\]
In some cases, we may want to regularize with a more general
norm $\|x\|_M^2 = x^T M x$ where $M$ is symmetric and positive
definite, which leads to the regularized equations
\[
  (A^T A + \lambda^2 M) x = A^T b.
\]
If we know of no particular problem structure in advance, the
standard choice of $M = I$ is a good default.

It is useful to compare the usual least squares solution to the
regularized solution via the SVD.  If $A = U \Sigma V^T$ is the
economy SVD, then
\begin{align*}
  x_{LS} &= V \Sigma^{-1} U^T b \\
  x_{Tik} &= V f(\Sigma)^{-1} U^T b
\end{align*}
where
\[
  f(\sigma)^{-1} = \frac{\sigma}{\sigma^2 + \lambda^2}.
\]
This {\em filter} of the inverse singular values affects the larger
singular values only slightly, but damps the effect of very small
singular values.

\section{Truncated SVD}

The Tikhonov filter reduces the effect of small singular values on
the solution, but it does not eliminate that effect.  By contrast,
the {\em truncated SVD} approach uses the filter
\[
f(z) =
\begin{cases}
  z, & z > \sigma_{\min} \\
  \infty, & \mbox{otherwise}.
\end{cases}
\]
In other words, in the truncated SVD approach, we use
\[
  x = V_k \Sigma_k^{-1} U_k^T b
\]
where $U_k$ and $V_k$ represent the leading $k$ columns of $U$ and
$V$, respectively, while $\Sigma_k$ is the diagonal matrix consisting
of the $k$ largest singular values.

\section{$\ell^1$ and the lasso}

An alternative to Tikhonov regularization (based on a Euclidean norm
of the coefficient vector) is an $\ell^1$ regularized problem
\[
  \mbox{minimize } \|Ax-b\|^2 + \lambda \|x\|_1.
\]
This is sometimes known as the ``lasso'' approach.  The $\ell^1$
regularized problem has the property that the solutions tend to
become sparse as $\lambda$ becomes larger.  That is, the $\ell^1$
regularization effectively imposes a factor selection process like
that we saw in the pivoted QR approach.  Unlike the pivoted QR
approach, however, the $\ell^1$ regularized solution cannot be
computed by one of the standard factorizations of numerical linear
algebra.  Instead, one treats it as a more general {\em convex
  optimization} problem.  We will discuss some approaches to the
solution of such problems later in the semester.

\section{Tradeoffs and tactics}

All four of the regularization approaches we have described are used
in practice, and each has something to recommend it.  The pivoted QR
approach is relatively inexpensive, and it results in a model that
depends on only a few factors.  If taking the measurements to compute
a prediction costs money --- or even costs storage or bandwidth for
the factor data! --- such a model may be to our advantage.  The
Tikhonov approach is likewise inexpensive, and has a nice Bayesian
interpretation (though we didn't talk about it).  The truncated SVD
approach involves the best approximation rank $k$ approximation to the
original factor matrix, and can be interpreted as finding the $k$ best
factors that are linear combinations of the original measurements.
The $\ell_1$ approach again produces models with sparse coefficients;
but unlike QR with column pivoting, the $\ell_1$ regularized solutions
incorporate information about the vector $b$ along with the matrix $A$.

So which regularization approach should one use?  In terms of
prediction quality, all can provide a reasonable deterrent against
ill-posedness and overfitting due to highly correlated factors.  Also,
all of the methods described have a parameter (the number of retained
factors, or a penalty parameter $\lambda$) that governs the tradeoff
between how well-conditioned the fitting problem will be and the
increase in bias that naturally comes from looking at a smaller class
of models.  Choosing this tradeoff intelligently may be rather more
important than the specific choice of regularization strategy.  A
detailed discussion of how to make this tradeoff is beyond the scope
of the class; but we will see some of the computational tricks
involved in implementing specific strategies for choosing
regularization parameters before we are done.

\end{document}
